\subsection{Astrobiology Methods}
\label{sec:Astrobiology Methods}

\subsubsection{Pre-Flight Preparation}

The clean box, collection containers and tubing will be autoclaved. All tools used in the assembly of the clean box will either be autoclaved or soaked in a \SI{70}{\percent} ethanol solution inside of a clean room. Each person who enters the clean room will be garbed in a lab coat, goggles, hair net and latex gloves after thoroughly washing their hands in a \SI{70}{\percent} ethanol solution. 

We will varry the glycerol concentration and increase the amount of medium within the assembly.  The \SI{15}{\percent}, \SI{25}{\percent}, and \SI{30}{\percent} glycerol solutions will be poured into each container, the lid will then be sealed with silicone gasket maker along with the tubing inserted and gasket sealed into the container lids. Each lid will have two holes, one that leads to the inside of the clean box to allow for pressure to be released from inside the container and outgassed through the valve that will be embedded in the box, while the other hole will be passed through the clean box lid to allow the tubing to connect to the pump - solenoid system only in the case of the control tubing. The lid to the clean box willl be sealed with silicone gasket maker, the box will be mounted onto the payload and the tube from the control container will be clamped to the dedicated control solenoid, while the sample collecting tube will be passed through the other solenoid and connected to the pump. A final piece of tubing will be connected from the intake valve on the pumps to the outside of the payload, after a \SI{70}{\percent} ethanol solution is ran through the pump several times. The end of the tubing will connect to a mechanism that will isolate the inner tubing until float conditions are reached. The payload will then be closed and remain  in the clean room until it is ready for transport flight.

\subsubsection{Post Flight Procedures}

Once the payload is retrieved, the intact clean box needs to be removed and placed inside of a cooler with ice to be then transported to The University of Houston and placed in cold storage at \SI{4}{\celsius}. All equipment used in the filtration process will be either autoclaved or taken from previously unopened sanitized packaging. The autoclaved, pre-sanitized items and the clean box will then be washed in a \SI{70}{\percent} ethanol solution before they are placed inside a SterilGARD e3 Class II Biological Safety Cabinet (the Cabinet). The Cabinet has a laminar flow air barrier and UV lights built into the ceiling for decontaminating the workspace prior to use. A portion of both the control and sample collection solutions will be vacuum filtered through a Fluropore membrane filter (\SI{13}{\milli\meter}; \num{0.22} micron) to collect specimens on the filter surface.  The filters will then packaged for in-house 16S ribosomal RNA sequencing.  In addition, the remaining portion of the glycerol solutions will be used on various culturing media.

%\subsubsection{Ribosomal RNA Sequencing and Data Analysis}

%We sent our filtered control and experimental samples to RTL Genomics in Lubbock, Texas for ribosomal RNA sequencing and data analysis. We selected the 926wF  (``AAACTYAAAKGAATTGRCGG'') and 1392R (``ACGGGCGGTGTGTRC***'') primers for the sequencing procedure. These primers are designed to amplify 16S RNA from bacterial, archaeal and eukaryotic ``universal'' samples. The samples were amplified using a two step PCR procedure~\cite{lemon}. Samples were sequenced on an Illumina MiSeq~\cite{illumina} 2x300 flow cell at 10~pM and sequenced at RTL Genomics using standard sequencing procedures~\cite{Microbial rRNA sequencing}.

