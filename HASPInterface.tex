\section{HASP Interface}
\label{sec:Hasp-Interface}

\subsection{Interfacing With HASP: Serial Uplink and Downlink}
During the duration of our flight the serial uplink commands will be used to configure the collection parameters of the MiniPIX radiation detector. When a command is received from the mission control team, it will be transmitted straight through the DB9 connection into the RPI3 through an RS232 to ttl converter. Then the monitoring script on the RPI3 will verify its content and perform the appropriate action defined in Table~\ref{tab:All-Commands}. The baud rate for serial communication will be set to 4800 bits per second ($bits/sec$). 


\begin{table}[!ht]
\centering
\caption{Table of All Uplink Command} 
\label{tab:All-Commands}
\bigskip
\begin{tabular}{|c|c|c|c|}
\hline
\multicolumn{1}{|c|}{\bfseries Command} & \multicolumn{1}{c|}{\bfseries HEX Uplink Command} & \multicolumn{1}{c|}{\bfseries HEX Uplink Argument}\\
\hline
    Begin Acquisitions 		& 0x01	 & N/A			    	    \\ \hline 
    Stop Acquisitions 		& 0x02	 & N/A			            \\ \hline
    Set Shutter Time		& 0x03   & Shutter Time in seconds (1-255)  \\ \hline %two pumps now
    Change Acquisition Mode    	& 0x04 	 & HEX Mode Identifier 	            \\ \hline
    Device Reset		& 0x05 	 & N/A				    \\ \hline
    
\end{tabular}
\medskip
\end{table}

\begin{table}[!ht]
\centering
\caption{Table of Acquisition Modes} 
\label{tab:Acq-Commands}
\bigskip
\begin{tabular}{|c|c|c|}
\hline
\multicolumn{1}{|c|}{\bfseries Acquisition Mode} & \multicolumn{1}{c|}{\bfseries Hex Mode Identifier} & \multicolumn{1}{c|}{\bfseries Description}\\
\hline
    Automatic Shutter Time	& 0x01   & Shutter time determined automatically based on particle flux   \\ \hline %two pumps now
    Fixed Shutter Time    	& 0x02 	 & Shutter time fixed to set shutter time 	     		  \\ \hline
    Counting Mode    		& 0x03 	 & Configure MiniPIX to measure counts on the detector only 	  \\ \hline
    ToT Mode			& 0x04	 & Configure MiniPIX to measure time over threshold values	  \\ \hline
    ToA Mode			& 0x05 	 & Configure MiniPIX to measure time of arrival 	     	  \\ \hline
\end{tabular}
\medskip
\end{table}


\subsection{Interfacing with HASP: EDAC and DB9 Connections}

Our wiring will not change much from the last mission~\cite{SORA}.  From the EDAC pins, we will join two of the \SI{30}{\volt} leads together in parallel to supply power to the central power supply.  Likewise, two of the ground leads will be used to ground the supply.  A pair of \SI{30}{\volt} and ground leads will also be used as input to the relay circuit in order to supply power to the pumps which require at least \SI{24}{\volt} to operate.  The last pair of \SI{30}{\volt} and ground leads will be used as input to the punch and hold circuit mentioned in the previous section to power the solenoids which require at least \SI{20}{\volt} to operate.  Two temperature sensors will be connected by both analog leads from the EDAC connection.  The corresponding ground pins will be used to ground them.  These two sensors will be placed on each of the pumps.

We will use discrete commands in this mission.  Two of the commands will be used to turn the astrobioly system on and off.  This is only in case of emergency if we notice that the system is not responding to our serial commands.  The other two discrete channels will be used to turn on and off the power of RESU in case that we notice an issue.

\begin{table}[!ht]
\centering
\caption{Table of all discrete commands to be used during flight} 
\label{tab:Dis-Commands}
\bigskip
\begin{tabular}{|c|c|c|c|}
\hline
\multicolumn{1}{|c|}{\bfseries Command} & \multicolumn{1}{c|}{\bfseries Purpose} &  \multicolumn{1}{c|}{\bfseries EDAC Pin} & \multicolumn{1}{c|}{\bfseries Description} \\
\hline
    Discrete 1     	& Astro. System ON 	& f	 & Initiates the pumps and collection systems   \\ \hline 
    Discrete 2    	& Astro. System OFF 	& n	 &  Shuts down the pumps and causes collection systems to retract  \\ \hline
    Discrete 3  	& RESU On 	& h	 & Powers up RESU   \\ \hline 
    Discrete 4 		& RESU OFF 	& p	 & Shuts down RESU   \\ \hline
\end{tabular}
\medskip
\end{table}

