\section{Thermal Control Plan}
\label{sec:TCP}
Based on our previous mission~\cite{SORA}, we opted to have a similar thermal control plan but with a few changes.  To simplify the payload construction and control, we opted for an manual temperature monitoring and control system.  

Each device will have a temperature sensor and we will actively monitor the temperatures based on the downlink packet information.  If we notice an upwards trend or downwards trend in temperature change, we will adjust the heaters to turn on/off.  For other devices without heaters, we will turn them off if they are in danger of overheating.

In the case of our payload experiencing colder temperatures, all of our devices will be able to operate beyond expected temperatures on the flight.  For the pumps, they will need to have heaters installed to maintain operation.

The temperature will be checked and recorded periodically for all devices.  Based on downlinked values, we will make decisions from the ground to manually manage our payload.

In order to manage the temperature of the  MiniPIX will be fitted with the same heatsink configuration as our 2017 flight. The internal temperature of the device will be recorded and downlinked to the flight control team at regular intervals. If the device is beginning to experience temperatures beyond what we would consider safe (above \SI{70}{\celsius}) we can either reset the RPI3 controlling the MiniPIX or fully shutdown the entire radiation subsystem.

%\begin{figure}[!htb]
%\begin{center}
%\includegraphics[width=0.5\textwidth]{./Figures/Test.jpg}
%\caption{This is a test figure.
%}
%\label{fig:Test} 
%\end{center}
%\end{figure}
