%\clearpage
\subsection{RESU Methods}
\label{sec:RESU}


\subsubsection{Overview}

RESU will again take charge of many components that work in tandem to accurately describe the flight mission through data.  As mentioned briefly before, the sensors will transmit their readings to the RP3 via serial USB.  The RP3 would then take those strings and write them to a CSV file that corresponded to that sensor.  The data strings are timestamped using the Arduino's RTC sensor.  The biggest challenge will be keeping the power consumption within bounds while running all three experiments
in our previous mission with the addition of another pump.  But now with a more developed version of RESU, this flight will once again test the limits of our prototypes.

\subsubsection{The Sensors}

Each sensor will be digital or analog in nature.  Digital sensors typically provide digital filtering within themselves and are temperature compensated for the purpose of this mission. 
   
To mitigate the risk of component failure due to overheating or a decrease in temperature, the temperature sensors will be placed on the pumps, electronics bay, and MiniPIX, helping determine if the heaters need to be switched on or off.  In the case of overheating, the sensors would be used to determine whether the pumps must be shut off.  These sensors operate in the \SIrange{-40}{125}{\celsius} range and draw less than \SI{0.5}{\micro\ampere} of current, each.  Temperature readings will be recorded every \SIrange{3}{4}{\second}.

A pressure sensor will be placed inside the electronics container to monitor the pressure levels within the payload.  The sensor operates in the \SIrange{0}{14}{\bara} range, within a temperature range of \SIrange{-40}{85}{\celsius}, and draws \SI{1}{\micro\ampere} of current.  The pressure will be measured in the \SIrange{0}{14000}{\milli\bara} range, at a resolution of \SI{1}{\milli\bara} every \SIrange{3}{4}{\second}.

One humidity sensor will be placed inside at the electronics plate.  The sensor operates in a range of \SIrange{0}{100}{\percent} relative humidity and draws \SI{200}{\micro\ampere} of current.  Humidity will be measured at a resolution of \SI{5}{\percent} every \SIrange{3}{4}{\second}.

The Real-Time Clocks (RTC) records time in a $HH$:$MM$:$SS$ and $MM/DD/YYYY$ format.  The sensor operates in the \SIrange{-40}{85}{\celsius} range and consumes \SIrange{400}{500}{\micro\ampere} of current.  The timestamps  will be used for log when a specific data point is collected by each sensor in the Arduino.  The RP3  uses an RTC to help log for both the temperatures of the MiniPIX and RP3 respectfully.

The inertial measuring unit (IMU) is a combination sensor that reports orientation information.  It operates in a \SIrange{-40}{125}{\celsius} range and consumes \SI{450}{\micro\ampere}, \SI{3.2}{\milli\ampere}, and \SI{280}{\micro\ampere} of current, for the accelerometer, gyroscope, and magnetometer, respectively.  It uses the best qualities of each sensor to compliment the flaws in data acquisition over long time frames.  Data will be collected every \SIrange{5}{7}{\second}.
 
 %UVC radiation was measured by three identical sensors.  They supported light ranging from \SIrange{210}{380}{\watt\per\square\meter} and were located outside the roof of the payload.  These sensors were analog so their readings were voltages induced by incident rays on the surface of each sensor. To stabilize the voltage readings from each sensor, a \SI{3300}{\micro\farad} capacitor was placed across the ground and analog out pins of the sensor.  Data was collected every \SIrange{3}{4}{\second}.


\subsubsection{Powering It All Up}

In order to stay within the power constraints, a robust power supply will need to  handle all the components of the payload.  The power supply we will be using is the PPM-DC-ATX-P by WinSystems INC.  It offered the desired number of \SI{+5}{\volt} and \SI{+12}{\volt} outputs needed to power the payload's electronics.  This power supply could effectively take \SI{+30}{\volt} and step it down to two \SI{+12}{\volt} and two \SI{+5 }{\volt} outputs.  One of the \SI{+12}{\volt} outputs goes to the Arduino since it can step down to the appropriate voltages internally while the other goes to a PWM motor for the solenoid.  One of the \SI{+5 }{\volt} outputs powers two analog sensors that will be sent to HASP through the EDAC connection (more on that in the next sections).  The remaining \SI{+5 }{\volt} output is converted to a USB power cable for the RP3.  The power supply also has four ground outputs that will be used by each respective component. 

\subsubsection{Uplink Commands: Relay System for Pumps and Heaters}

When a command is received from the mission control team, it will be transmitted straight through the DB9 connection into the Arduino which will either accept or decline it based on our state machine.  If accepted, the state machine would then interpret which hexadecimal command was sent and would then send a logical HIGH pulse to the relay which subsequently turns a system on or off.  Table~\ref{tab:All-Commands} lists all the commands to be used during flight.

\begin{table}[!ht]
\centering
\caption{Table of All Uplink Commands Used During Flight} 
\label{tab:All-Commands}
\bigskip
\begin{tabular}{|c|c|c|c|}
\hline
\multicolumn{1}{|c|}{\bfseries Command} & \multicolumn{1}{c|}{\bfseries HEX Uplink Command} &  \multicolumn{1}{c|}{\bfseries ASCII Uplink Command} & \multicolumn{1}{c|}{\bfseries Expected Current Consumption} \\
\hline
    Pumps On     	& 0x65FF 	& e	 & \SI{1.63}{\ampere}   \\ \hline %two pumps now
    Pumps Off    	& 0x71FF 	& q	 & \SI{0.29}{\ampere}    \\ \hline
    Heaters On  	& 0x68FF 	& h	 & \SI{0.65}{\ampere}    \\ \hline %two heaters now
    Heaters Off 	& 0x76FF 	& v	 & \SI{0.29}{\ampere}    \\ \hline
\end{tabular}
\medskip
\end{table}

Serial uplink commands allow for 4-byte hexadecimal strings.  Since none of our processes require more than a single parameter, we elected to leave the second parameter as 0xFF.  Note that every ``disengage'' results in our payload returning to nominal current which is about \SI{0.29}{\ampere}.  The ``pumps on'' command executed a sequence of events. After receiving the command, two normally closed solenoids would open and then after one second the pumps would begin the collection process.  The solenoids does have an initial peak current of \SI{1.2}{\ampere} each and to suppress it, a punch-and-hold circuit will be implemented using a Pololu G2 High Power Motor Driver.  It serves the purpose of reducing the current consumption from the solenoid to stay within the power budget.  The heaters are there to make sure the pumps can turn on in the extreme cold temperatures.  

\subsubsection{Interfacing with HASP: EDAC and DB9 Connections}

Our wiring will not change much from the last mission~\cite{SORA}.  From the EDAC pins, we will join two of the \SI{30}{\volt} leads together in parallel to supply power to the central power supply.  Likewise, two of the ground leads will be used to ground the supply.  A pair of \SI{30}{\volt} and ground leads will also be used as input to the relay circuit in order to supply power to the pumps which require at least \SI{24}{\volt} to operate.  The last pair of \SI{30}{\volt} and ground leads will be used as input to the punch and hold circuit mentioned in the previous section to power the solenoids which require at least \SI{20}{\volt} to operate.  Two temperature sensors will be connected by both analog leads from the EDAC connection.  The corresponding ground pins will be used to ground them.  These two sensors will be placed on the pumps - one sensor per pump.  

We will use discrete commands in this mission.  Two of the commands will be used to turn the astrobioly system on and off.  This is only in case of emergency if we notice that the system is not responding to our serial commands.  The other two discrete channels will be used to turn on and off the power of RESU in case that we notice an issue.

\begin{table}[!ht]
\centering
\caption{Table of all discrete commands to be used during flight} 
\label{tab:Dis-Commands}
\bigskip
\begin{tabular}{|c|c|c|c|}
\hline
\multicolumn{1}{|c|}{\bfseries Command} & \multicolumn{1}{c|}{\bfseries Purpose} &  \multicolumn{1}{c|}{\bfseries EDAC Pin} & \multicolumn{1}{c|}{\bfseries Description} \\
\hline
    Discrete 1     	& Astro. System ON 	& f	 & Initiates the pumps and collection systems   \\ \hline 
    Discrete 2    	& Astro. System OFF 	& n	 &  Shuts down the pumps and causes collection systems to retract  \\ \hline
    Discrete 3  	& RESU On 	& h	 & Powers up RESU   \\ \hline 
    Discrete 4 		& RESU OFF 	& p	 & Shuts down RESU   \\ \hline
\end{tabular}
\medskip
\end{table}




As for the DB9 connection, we will again use it to send serial uplink and downlink commands to better our flight operation, as well as to ensure the best possible experiment takes place while at float.  The Tx and Rx leads will be connected to the Arduino which then reads in the serial signal for reading and writing.  Each uplink command will be verified by a state machine that processes each string of \num{32} bits.  A Maxim 232 integrated circuit will be used in between the transmission and receiving leads to and from the Arduino which handles the input of the serial communication, ensuring proper integration with HASP voltages and Arduino voltages.  The baud rate for serial communication will be set to 4800 bits per second ($bits/sec$). 
%Listing~\ref{Downlinks} is a sample downlink data packet received during our previous flight.

%\lstset{basicstyle=\small, numbers=left, xleftmargin=2em, frame=tb, label = Downlinks, framexleftmargin=1.5em}
%\begin{lstlisting}[caption = Sample of downlinked data packets ID: 15667 - 15670 from SORA 2017~\cite{SORA}.]
%...
%begin_packet
%15667,-21.05,-0.33,35.62,-1.00,0.12,0.25,-0.71,0.63,9.30,-30.25,-26.69,-60.88
%3.14,2.86,0.00,0.00,0.60,9:29:39 9/4/17
%end_packet
%begin_packet
%15668,-20.95,-0.54,35.62,0.75,0.06,-0.69,0.25,-2.00,9.67,-28.37,-28.06,-60.06
%3.80,3.33,0.00,0.00,0.60,9:29:42 9/4/17
%end_packet
%begin_packet
%15669,-21.15,-0.54,35.62,0.81,-0.06,-0.56,1.22,-2.13,9.86,-29.56,-27.75,-57.88
%1.86,2.86,0.01,0.02,0.61,9:29:45 9/4/17
%end_packet
%begin_packet
%15670,-21.15,-0.54,35.62,-1.00,0.00,0.12,-0.55,0.73,9.37,-27.37,-30.00,-58.75
%1.86,3.96,0.00,0.00,0.61,9:29:48 9/4/17
%end_packet
%...
%\end{lstlisting}
%\medskip
%
%Each packet of data will be delimited by keywords \verb|begin_packet| and \verb|end_packet| so that parsing each file be easier.  During our previous flight, these data packets were crucial status updates to the state of our payload.  We will once again use them for the same purpose to keep a close eye on our payload.
%
%Data packets will contain information about the readings from each sensor.  The first integer represents the ID of the packet.  Following the packet ID are the three temperature readings, gyroscope $x, y, z$ values, accelerometer $x, y, z$ values, magnetometer $x, y, z$ values, pressure readings, humidity readings, UV \#1, \#2, \#3 voltage readings, and finally the timestamp in $HH$:$MM$:$SS$ $MM$/$DD$/$YY$ at which the packet was written.  
%
%In addition to downlinking sensor data, we also want to downlink serial uplink commands as shown in listing \ref{Uplinks}.  
%
%\lstset{basicstyle=\small, numbers=left, frame=tb, linewidth=11.5cm, xleftmargin=.4\textwidth, label = Uplinks}
%\begin{lstlisting}[caption = Sample of received uplink commands in downlinked packets in SORA 2017~\cite{SORA}]
%...
%1
%2
%71
%FFFFFFFF
%3
%D
%A
%Received Command: 71
%...
%\end{lstlisting}
%\medskip
%
%Each line within a received uplink command represents the string of bytes that will be read by our payload.  The last line of the listing shows what command will be parsed and processed by RESU.  Table \ref{tab:All-Commands} shows all the possible commands that we are expecting to process. 
