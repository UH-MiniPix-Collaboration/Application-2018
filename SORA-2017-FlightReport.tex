\documentclass[aps,superscriptaddress,floatfix,nofootinbib,showpacs,amsmath,amssymb,altaffilletter,floatfix,onecolumn]{revtex4-1}
%add "rmp" within document class to make it double column and "twocolumn"
%---

%HASP Variables
\newcommand{\MPThreshold}{\SI{4.00}{\kilo\eV}}
%--- Packages
\usepackage[colorlinks=true,pdfstartview=FitV,linkcolor=blue,citecolor=blue,urlcolor=blue]{hyperref}
\usepackage[separate-uncertainty,retain-explicit-plus,per-mode=symbol,binary-units]{siunitx}
\usepackage{array,mathtools,amssymb,dcolumn}
\usepackage{amsmath}
\usepackage[below]{placeins}
\usepackage[table]{xcolor}
\usepackage{tikz}
\usepackage{afterpage}
\usepackage{lineno}
\usepackage{paralist}
\usepackage{listings}
\usepackage{array}
\usepackage[version=4]{mhchem}
\usepackage{multirow}
\usepackage{eurosym}
\usepackage{pagecolor}
\usepackage{fancyhdr}
%--- 1 inch margins
\usepackage{calc}
\setlength\textwidth{6.5in}
\setlength\textheight{9in}
\setlength\oddsidemargin{(\paperwidth-\textwidth)/2 - 1in} 
\setlength\evensidemargin{(\paperwidth-\textwidth)/2 - 1in} 
\setlength\topmargin{(\paperheight-\textheight-\headheight-\headsep-\footskip)/2 - 1in}
%--- Language
\lstset{language=C++,basicstyle=\ttfamily}
%--- Floats Placement
\setlength\textfloatsep{5pt}
\setlength\abovecaptionskip{5pt}
%--- Counters
\newcounter{mylistcounter}
%--- Text and References
\newcommand{\myrefs}[2]{\href{http://dx.doi.org/#2}{#1}}
\newcommand{\mref}[1]{\href{http://#1}{#1}}
\newcommand{\elog}[1]{\href{https://blackhole.lngs.infn.it/DS-50kg/#1}{#1}}
\newcommand{\mrefsec}[1]{\href{https://#1}{#1}}
\newcommand{\mrefs}[2]{\href{http://#2}{#1}}
\newcommand{\arxiv}[1]{\href{http://arxiv.org/abs/#1}{arxiv:#1}}
\newcommand{\grant}[2]{#1-#2}
\newcommand{\docdb}[1]{\href{http://darkside-docdb.fnal.gov:8080/cgi-bin/ShowDocument?docid=#1}{DarkSide DocDB \##1}}
\newcommand{\cmt}[2]{\indent{\tt \color{blue}#1: \color{red}#2}}
\newcommand{\chk}[1]{{\tt \color{red}To be checked:~#1}}
\newcommand{\fchk}[1]{{\tt \color{red}Figure to be replaced:~#1}}
\newcommand{\event}[2]{{\tt Event\# #1, Run\# #2}}
\newcommand{\minitab}[3]{\begin{tabular}{@{}#1@{}}{#2}\\{#3}\end{tabular}}
%--- Software Packages
\newcommand{\FLUKA}{\mbox{FLUKA}}
\newcommand{\Geant}{\mbox{Geant4}}
\newcommand{\GFDS}{\mbox{G4DS}}
\newcommand{\SOURCES}{\mbox{SOURCES4A}}
\newcommand{\TALYS}{\mbox{TALYS}}
\newcommand{\SRIM}{\mbox{SRIM}}
\newcommand{\LabVIEW}{\mbox{NI LabVIEW}}
\newcommand{\CERNRoot}{\mbox{Root}}
%--- Functions
\newcommand{\logten}{\ensuremath{\log_{10}}}
%--- Units
\DeclareSIUnit\c{\mbox{$c$}}
\DeclareSIUnit\magn{\mbox{$\times$}}
\DeclareSIUnit\min{min}
\DeclareSIUnit\week{week}
\DeclareSIUnit\year{yr}
\DeclareSIUnit\years{years}
\DeclareSIUnit\yr{yr}
\DeclareSIUnit\standard{std}
\DeclareSIUnit\str{sr}
\DeclareSIUnit\ppm{ppm}
\DeclareSIUnit\ppb{ppb}
\DeclareSIUnit\ppt{ppt}
\DeclareSIUnit\pe{PE}
\DeclareSIUnit\spe{SPE}
\DeclareSIUnit\ev{events}
\DeclareSIUnit\ct{counts}
\DeclareSIUnit\neutron{\mbox{$n$}}
\DeclareSIUnit\smp{samples}
\DeclareSIUnit\Sample{S}
\DeclareSIUnit\ch{ch}
\DeclareSIUnit\hit{hit}
\DeclareSIUnit\hits{hits}
\DeclareSIUnit\bin{(\mbox{5-PE}~bin)}
\DeclareSIUnit\sgm{\mbox{$\sigma$}}
\DeclareSIUnit\rms{RMS}
\DeclareSIUnit\keVr{\mbox{keV$_{\rm nr}$}}
\DeclareSIUnit\keVee{\mbox{keV$_{e{\rm e}}$}}
\DeclareSIUnit\ph{photons}
\DeclareSIUnit\pm{PMT}
\DeclareSIUnit\inch{''}
\DeclareSIUnit\feet{'}
\DeclareSIUnit\bit{bit}
\DeclareSIUnit\sample{samples}
\DeclareSIUnit\barn{barn}
\DeclareSIUnit\bara{bar}
\DeclareSIUnit\barg{barg}
\DeclareSIUnit\mlardepth{\mbox(meter~of~\LAr~depth)}
\DeclareSIUnit\Curie{Ci}
\DeclareSIUnit\psi{psi}
\DeclareSIUnit\parsec{pc}
\DeclareSIUnit\liveday{\mbox{live-days}}
\DeclareSIUnit\days{\mbox{days}}
\DeclareSIUnit\day{\mbox{day}}
\DeclareSIUnit\miles{\mbox{miles}}
\DeclareSIUnit\degreeC{\mbox{$^{\circ}$C}}
\DeclareSIUnit\electron{\mbox{$e^-$}}
\DeclareSIUnit\Euro{\mbox{\euro}}
\DeclareSIUnit\cph{cph}
\DeclareSIUnit\neq{neq}
\DeclareSIUnit\Gray{Gy}
%--- Energies, Branching Ratios, and Abundances
\newcommand{\BR}{\mbox{BR}}
\newcommand{\EC}{\mbox{EC}}
\newcommand{\PositronAnnihilationGammaEnergy}{\SI{511}{\keV}}
\newcommand{\PbXRayEnergy}{\SI{46}{\keV}}
\newcommand{\HOneNeutronCaptureGammaEnergy}{\SI{2.2}{\MeV}}
\newcommand{\LiSixNaturalAbundance}{\SI{7.5}{\percent}}
\newcommand{\LiSixNeutronCaptureCrossSection}{\SI{941}{\barn}}
\newcommand{\LiSixNeutronCaptureTritonEnergy}{\SI{2.73}{\MeV}}
\newcommand{\LiSixNeutronCaptureAlphaEnergy}{\SI{2.05}{\MeV}}
\newcommand{\LiSixNeutronCaptureTritonAlphaQuenchedEnergy}{\SIrange[range-units=single]{400}{500}{\keVee}}
\newcommand{\BTenNaturalAbundance}{\SI{20}{\percent}}
\newcommand{\BTenNeutronCaptureCrossSection}{\SI{3840}{\barn}}
\newcommand{\BTenNeutronCaptureGroundDecayBR}{\SI{6.4}{\percent}}
\newcommand{\BTenNeutronCaptureGroundDecayAlphaEnergy}{\SI{1775}{\keV}}
\newcommand{\BTenNeutronCaptureExcitedDecayBR}{\SI{93.6}{\percent}}
\newcommand{\BTenNeutronCaptureExcitedDecayGammaEnergy}{\SI{478}{\keV}}
\newcommand{\BTenNeutronCaptureExcitedDecayAlphaEnergy}{\SI{1471}{\keV}}
\newcommand{\BTenNeutronCaptureExcitedDecayAlphaQuenchedEnergy}{\SIrange[range-units=single]{30}{35}{\keVee}}
\newcommand{\BTenNeutronCaptureExcitedDecayAlphaPE}{\SIrange[range-units=single]{25}{35}{\pe}}
\newcommand{\COneFourQValue}{\SI{156}{\keV}}
\newcommand{\CoFiveSevenQValue}{\SI{122}{\keV}}
\newcommand{\BaOneThreeThreeQValue}{\SI{356}{\keV}}
\newcommand{\CsOneThreeSevenQValue}{\SI{662}{\keV}}
\newcommand{\ArThreeSevenDecay}{\EC}
\newcommand{\ArThreeSevenBR}{\SI{100}{\percent}}
\newcommand{\ArThreeSevenQValue}{\SI{2.7}{\keV}}
\newcommand{\ArThreeSevenMeanLife}{\SI{50.51(3)}{\day}}
\newcommand{\ArThreeSevenHalfLife}{\SI{35.04}{\day}}
\newcommand{\ArThreeSevenKOneBR}{\SI{81.5}{\percent}}
\newcommand{\ArThreeSevenKTwoToFourBR}{\SI{8.7}{\percent}}
\newcommand{\ArThreeSevenKCaptureXRaysEnergy}{\SI{2.82}{\keV}}
\newcommand{\ArThreeNineQValue}{\SI{565}{\keV}}
\newcommand{\ArThreeNineMeanLife}{\SI{388}{\year}}
\newcommand{\RbEightThreeMeanLife}{\SI{124.4}{\day}}
\newcommand{\RbEightFiveMGammaEnergy}{\SI{514}{\keV}}
\newcommand{\RbEightFiveMMeanLife}{\SI{1.464}{\micro\s}}
\newcommand{\KrEightThreeQValue}{\SI{41.5}{\keV}}
\newcommand{\KrEightThreeMOneMeanLife}{\SI{2.64}{\hour}}
\newcommand{\KrEightThreeMOneECEnergy}{\SI{32.1}{\keV}}
\newcommand{\KrEightThreeMTwoMeanLife}{\SI{222}{\nano\second}}
\newcommand{\KrEightThreeMTwoECEnergy}{\SI{9.4}{\keV}}
\newcommand{\KrEightThreeMOneTwoECEnergy}{\SI{41.5}{\keV}}
\newcommand{\KrEightFiveGroundDecayQValue}{\SI{687}{\keV}}
\newcommand{\KrEightFiveExcitedDecayBR}{\SI{0.43}{\percent}}
\newcommand{\KrEightFiveExcitedDecayQValue}{\SI{173}{\keV}}
\newcommand{\GdNatNeutronCaptureCrossSection}{\SI{48890}{\barn}}
\newcommand{\PbTwoOneZeroHalfLife}{\SI{22.3}{\yr}}
\newcommand{\PbTwoOneZeroMeanLife}{\SI{32.0}{\yr}}
\newcommand{\PoTwoOneZeroAlphaEnergy}{\SI{5.3}{\MeV}}
\newcommand{\PoTwoOneTwoAlphaEnergy}{\SI{8.78}{\MeV}}
\newcommand{\BiTwoOneTwoAlphaOneEnergy}{\SI{6.09}{\MeV}}
\newcommand{\BiTwoOneTwoAlphaTwoEnergy}{\SI{6.05}{\MeV}}
\newcommand{\BiTwoOneTwoHalfLife}{\SI{10.6}{\hour}}
\newcommand{\PoTwoOneSixAlphaEnergy}{\SI{6.78}{\MeV}}
\newcommand{\RnTwoTwoZeroHalfLife}{\SI{56}{\second}}
\newcommand{\RnTwoTwoZeroAlphaEnergy}{\SI{6.29}{\MeV}}
\newcommand{\RnTwoTwoTwoHalfLife}{\SI{3.8}{\day}}
\newcommand{\RaTwoTwoFourHalfLife}{\SI{3.6}{\day}}
\newcommand{\ThTwoTwoEightHalfLife}{\SI{1.9}{\yr}}
\newcommand{\AmTwoFourGammaOneEnergy}{\SI{59.5}{\keV}}
\newcommand{\AmTwoFourOneGammaTwoBR}{\SI{56}{\percent}}
\newcommand{\AmBeGammaEnergy}{\SI{4.4}{\MeV}}
\newcommand{\AmBe}{\ce{^241AmBe}}
\newcommand{\AmC}{\ce{^241Am^13C}}
\newcommand{\AmCNeutronEnergy}{\SI{4}{\MeV}}
\newcommand{\DD}{\ce{^2D}-\ce{^2D}}
\newcommand{\DDNeutronEnergy}{\SI{2.45}{\MeV}}
\newcommand{\AArArThreeNineOverArFourZeroRatio}{\num{8E-16}}
\newcommand{\AArArThreeNineActivity}{\SI{1}{\becquerel\per\kg}}
\newcommand{\NeutronsPerChainDecayUTh}{\numrange{E-5}{E-7}}
\newcommand{\LArRadiogenicNeutronInteractionLength}{\SI{~10}{\cm}}
%--- Cosmology
\newcommand{\LCDM}{\mbox{$\Lambda$CDM}}
%--- Solar Neutrinos
\newcommand{\PP}{\mbox{$pp$}}
\newcommand{\PEP}{\mbox{$pep$}}
\newcommand{\CNO}{\mbox{CNO}}
%--- Names
\newcommand{\DS}{\mbox{DarkSide}}
\newcommand{\DSt}{\mbox{DarkSide-10}}
\newcommand{\DSf}{\mbox{DarkSide-50}}
\newcommand{\DSp}{\mbox{DarkSide-Proto}}
\newcommand{\DSk}{\mbox{DarkSide-20k}}
\newcommand{\DSs}{\mbox{DS}}
\newcommand{\DSts}{\mbox{DS-10}}
\newcommand{\DSfs}{\mbox{DS-50}}
\newcommand{\DSps}{\mbox{DS-Proto}}
\newcommand{\DSks}{\mbox{DS-20k}}
\newcommand{\DSCollaborators}{\num{\sim~300}}
\newcommand{\DSInstitutes}{\num{\sim~70}}
\newcommand{\DSCountries}{\num{15}}
\newcommand{\GADMC}{\mbox{GADMC}}
\newcommand{\DEAP}{\mbox{DEAP-3600}}
\newcommand{\mCLEAN}{\mbox{MiniCLEAN}}
\newcommand{\ArDM}{\mbox{ArDM}}
\newcommand{\Argo}{\mbox{Argo}}
\newcommand{\ThreeDPi}{\mbox{3D$\pi$}}
\newcommand{\FBP}{\mbox{FBP}}
\newcommand{\BX}{\mbox{Borexino}}
\newcommand{\SNO}{\mbox{SNO}}
\newcommand{\SCENE}{\mbox{SCENE}}
\newcommand{\ReD}{\mbox{ReD}}
\newcommand{\ARIS}{\mbox{ARIS}}
\newcommand{\Urania}{\mbox{Urania}}
\newcommand{\Aria}{\mbox{Aria}}
\newcommand{\Seruci}{\mbox{Seruci}}
\newcommand{\SeruciZero}{\mbox{Seruci-0}}
\newcommand{\SeruciOne}{\mbox{Seruci-I}}
\newcommand{\SeruciTwo}{\mbox{Seruci-II}}
\newcommand{\LOGAN}{\mbox{LOGAN}}
\newcommand{\DART}{\mbox{DART}}
\newcommand{\MAWG}{M\&A\,WG}
\newcommand{\CTF}{\mbox{CTF}}
\newcommand{\WBS}{\mbox{WBS}}
\newcommand{\CROne}{\mbox{CR1}}
\newcommand{\CRH}{\mbox{CRH}}
\newcommand{\LSV}{\mbox{LSV}}
\newcommand{\WCV}{\mbox{WCV}}
\newcommand{\wt}{\mbox{WT}}
\newcommand{\TPC}{\mbox{TPC}}
\newcommand{\TPCs}{\mbox{TPCs}}
\newcommand{\LArTPC}{\mbox{LAr~TPC}}
\newcommand{\LArTPCs}{\mbox{LAr~TPCs}}
\newcommand{\calis}{\mbox{CALIS}}
\newcommand{\UV}{\mbox{UV}}
\newcommand{\NUV}{\mbox{NUV}}
\newcommand{\TMF}{\mbox{TMF}}
\newcommand{\PMT}{\mbox{PMT}}
\newcommand{\PMTs}{\mbox{\PMT s}}
\newcommand{\MCP}{\mbox{MCP}}
\newcommand{\MCPPMT}{\mbox{\MCP-\PMT}}
\newcommand{\MCPPMTs}{\mbox{\MCPPMT s}}
\newcommand{\SiPM}{\mbox{SiPM}}
\newcommand{\SiPMs}{\mbox{SiPMs}}
\newcommand{\RGBHd}{\mbox{RGB-HD}}
\newcommand{\RGBHdSf}{\mbox{RGB-HD-SF}}
\newcommand{\RGBHdSfHRq}{\mbox{RGB-HD-HR$_q$}}
\newcommand{\RGBHdSfLRq}{\mbox{RGB-HD-LR$_q$}}
\newcommand{\NUVHd}{\mbox{NUV-HD}}
\newcommand{\NUVHdSf}{\mbox{NUV-HD-SF}}
\newcommand{\NUVHdLf}{\mbox{NUV-HD-LF}}
\newcommand{\NUVHdSfHRq}{\mbox{NUV-HD-SF-HR$_q$}}
\newcommand{\NUVHdSfLRq}{\mbox{NUV-HD-SF-LR$_q$}}
\newcommand{\NUVHdLfHRq}{\mbox{NUV-HD-LF-HR$_q$}}
\newcommand{\NUVHdLfLRq}{\mbox{NUV-HD-LF-LR$_q$}}
\newcommand{\HRq}{\mbox{HR$_q$}}
\newcommand{\LRq}{\mbox{LR$_q$}}
\newcommand{\HD}{\mbox{HD}}
\newcommand{\CTE}{\mbox{CTE}}
\newcommand{\PCB}{\mbox{PCB}}
\newcommand{\PCBs}{\mbox{PCBs}}
\newcommand{\TSV}{\mbox{TSV}}
\newcommand{\TSVs}{\mbox{TSVs}}
\newcommand{\tile}{\mbox{tile}}
\newcommand{\tiles}{\mbox{tiles}}
\newcommand{\SPAD}{\mbox{SPAD}}
\newcommand{\SPADs}{\mbox{SPADs}}
\newcommand{\QE}{\mbox{QE}}
\newcommand{\PDE}{\mbox{PDE}}
\newcommand{\OV}{\mbox{OV}}
\newcommand{\LV}{\mbox{LV}}
\newcommand{\SCR}{\mbox{SCR}}
\newcommand{\DCR}{\mbox{DCR}}
\newcommand{\DiCT}{\mbox{DiCT}}
\newcommand{\DeCT}{\mbox{DeCT}}
\newcommand{\AP}{\mbox{AP}}
\newcommand{\DLED}{\mbox{DLED}}
\newcommand{\TCNR}{\mbox{TCNR}}
\newcommand{\TCNP}{\mbox{TCNP}}
\newcommand{\CMOS}{\mbox{CMOS}}
\newcommand{\HLST}{\mbox{HLST}}
\newcommand{\SQB}{\mbox{SQB}}
\newcommand{\SQBs}{\mbox{\SQB s}}
\newcommand{\TRB}{\mbox{TRB}}
\newcommand{\TRBs}{\mbox{\TRB s}}
\newcommand{\WIMP}{\mbox{WIMP}}
\newcommand{\WIMPs}{\mbox{\WIMP s}}
\newcommand{\MC}{\mbox{MC}}
\newcommand{\DAQ}{\mbox{DAQ}}
\newcommand{\ADC}{\mbox{ADC}}
\newcommand{\ADCs}{\mbox{\ADC s}}
\newcommand{\TDC}{\mbox{TDC}}
\newcommand{\TDCs}{\mbox{\TDC s}}
\newcommand{\artdaq}{\mbox{artdaq}}
\newcommand{\TMB}{\mbox{TMB}}
\newcommand{\PC}{\mbox{PC}}
\newcommand{\PCTMB}{\mbox{\PC-\TMB}}
\newcommand{\PPO}{\mbox{PPO}}
\newcommand{\BisMSB}{\mbox{BisMSB}}
\newcommand{\POPOP}{\mbox{POPOP}}
\newcommand{\DIN}{\mbox{DIN}}
\newcommand{\LAB}{\mbox{LAB}}
\newcommand{\PXE}{\mbox{PXE}}
\newcommand{\ITO}{\mbox{ITO}}
\newcommand{\PTFE}{\mbox{PTFE}}
\newcommand{\FEP}{\mbox{FEP}}
\newcommand{\TPB}{\mbox{TPB}}
\newcommand{\og}{\mbox{Operations Group}}
\newcommand{\VUV}{\mbox{VUV}}
\newcommand{\LAr}{\ce{LAr}}
\newcommand{\GAr}{\ce{GAr}}
\newcommand{\AAr}{\ce{AAr}}
\newcommand{\UAr}{\ce{UAr}}
\newcommand{\DAr}{\ce{DAr}}
\newcommand{\LRAr}{\ce{LRAr}}
\newcommand{\LIN}{\ce{LN2}}
\newcommand{\LXe}{\ce{LXe}}
\newcommand{\HPGe}{\mbox{HPGe}}
\newcommand{\HVFT}{\mbox{HVFT}}
\newcommand{\UHMWPE}{\mbox{UHMWPE}}
\newcommand{\TOF}{\mbox{TOF}}
\newcommand{\PET}{\mbox{PET}}
\newcommand{\TOFPET}{\mbox{\TOF-\PET}}
\newcommand{\PETCT}{\mbox{\PET /CT}}
\newcommand{\FDG}{\mbox{\ce{^18F}-FDG}}
\newcommand{\LOR}{\mbox{LOR}}
\newcommand{\CPU}{\mbox{CPU}}
\newcommand{\CT}{\mbox{CT}}
\newcommand{\CPUs}{\mbox{CPUs}}
\newcommand{\STEM}{\mbox{STEM}}
\newcommand{\ROI}{\mbox{ROI}}
\newcommand{\SPE}{\mbox{SPE}}
\newcommand{\SNR}{\mbox{SNR}}
\newcommand{\SNRAmp}{\mbox{SNR$_{\rm Amplitude}$}}
\newcommand{\SNRCharge}{\mbox{SNR$_{\rm Charge}$}}
\newcommand{\SNRFast}{\mbox{SNR$_{\rm Fast}$}}
\newcommand{\SNRFilter}{\mbox{SNR$_{\rm Filtered}$}}
\newcommand{\OA}{\mbox{OA}}
\newcommand{\GBP}{\mbox{GBP}}
\newcommand{\TIA}{\mbox{TIA}}
\newcommand{\TIAs}{\mbox{\TIA s}}
\newcommand{\FEB}{\mbox{FEB}}
\newcommand{\FEBs}{\mbox{\FEB s}}
\newcommand{\VFEB}{\mbox{VFEB}}
\newcommand{\VFEBs}{\mbox{\VFEB s}}
\newcommand{\VDCB}{\mbox{VDCB}}
\newcommand{\VDCBs}{\mbox{\VDCB s}}
\newcommand{\AVol}{\mbox{$A_{\tiny\rm Vol}$}}
\newcommand{\NoG}{\mbox{NG}}
\newcommand{\NoGTrue}{\mbox{$\NoG_{\rm True}$}}
\newcommand{\Tz}{\mbox{$T_z$}}
\newcommand{\eno}{\mbox{$e_n$}}
\newcommand{\enot}{\mbox{$e_n^2$}}
\newcommand{\ino}{\mbox{$i_n$}}
\newcommand{\inot}{\mbox{$i_n^2$}}
\newcommand{\Vno}{\mbox{$V_n$}}
\newcommand{\Vnot}{\mbox{$V_n^2$}}
\newcommand{\VnoTot}{\mbox{$V_n^{\rm Tot}$}}
\newcommand{\fRes}{\mbox{$f_{\rm Res}$}}
\newcommand{\FTDb}{\mbox{$f_{\rm 3Db}$}}
\newcommand{\FSDb}{\mbox{$f_{\rm 6Db}$}}
\newcommand{\TDb}{\SI{3}{\decibel}}
\newcommand{\SDb}{\SI{6}{\decibel}}
\newcommand{\PSDno}{\mbox{$\rm PSD_{n}$}}
\newcommand{\PSDnoReferencePower}{\SI{1}{\milli\watt}}
\newcommand{\NSF}{\mbox{NSF}}
\newcommand{\RAS}{\mbox{RAS}}
\newcommand{\GSSI}{\mbox{GSSI}}
\newcommand{\INFN}{\mbox{INFN}}
\newcommand{\MIUR}{\mbox{MIUR}}
\newcommand{\CERN}{\mbox{CERN}}
\newcommand{\LHC}{\mbox{LHC}}
\newcommand{\CFI}{\mbox{CFI}}
\newcommand{\NSERC}{\mbox{NSERC}}
\newcommand{\LNGS}{\mbox{LNGS}}
\newcommand{\LSC}{\mbox{LSC}}
\newcommand{\SNOLab}{\mbox{SNOLab}}
\newcommand{\TRIUMF}{\mbox{TRIUMF}}
\newcommand{\LNS}{\mbox{LNS}}
\newcommand{\DUNE}{\mbox{DUNE}}
\newcommand{\CNAF}{\mbox{CNAF}}
\newcommand{\PoliMi}{\mbox{PoliMi}}
\newcommand{\FBKTIFPA}{\mbox{FBK/TIFPA}}
\newcommand{\FBK}{\mbox{FBK}}
\newcommand{\MPD}{\mbox{MPD}}
\newcommand{\LFoundry}{\mbox{LFoundry}}
\newcommand{\SensL}{\mbox{SensL}}
\newcommand{\NOA}{\mbox{NOA}}
\newcommand{\LSO}{\mbox{LSO}}
\newcommand{\LYSO}{\mbox{LYSO}}
\newcommand{\CRT}{\mbox{CRT}}
\newcommand{\ASIC}{\mbox{ASIC}}
\newcommand{\ASICs}{\mbox{ASICs}}
\newcommand{\PSA}{\mbox{PSA}}
\newcommand{\UHV}{UHV}
\newcommand{\HalfLife}{\mbox{$\tau_{\frac{1}{2}}$}}
\newcommand{\ICPMS}{\mbox{ICP-MS}}
\newcommand{\LA}{\mbox{LA}}
\newcommand{\LAICPMS}{\mbox{\LA-\ICPMS}}
\newcommand{\BiPo}{\mbox{BiPo-3}}
\newcommand{\PEB}{\mbox{PEB}}
\newcommand{\PEBs}{\mbox{PEBs}}
\newcommand{\VO}{\mbox{VO}}
\newcommand{\LiDaR}{\mbox{LiDaR}}
\newcommand{\ArlonFiveFiveNT}{\mbox{Arlon~55-NT}}
\newcommand{\ArlonEightFiveN}{\mbox{Arlon~85-N}}
\newcommand{\ArlonEightFiveNT}{\mbox{Arlon~85-NT}}
\newcommand{\THERMOUNT}{\mbox{THERMOUNT\textregistered RT\textsuperscript{TM}}}
\newcommand{\Vikuiti}{\mbox{3M Vikuiti\textsuperscript{TM}}}
%--- Global Symbols
\newcommand{\OmegaNonBaryonicFraction}{\num{0.85}}
\newcommand{\OmegaNonBaryonicPercent}{\SI{85}{\percent}}
\newcommand{\RhoDMSymbol}{\mbox{$\rho_{\rm dm}$}}
\newcommand{\RhoDMValue}{\SI{0.3}{\GeV\per\square\c\per\cubic\cm}}
\newcommand{\VelocityNaughtSymbol}{\mbox{$v_0$}}
\newcommand{\VelocityNaughtValue}{\SI{220}{\km\per\s}}
\newcommand{\VelocityEarthSymbol}{\mbox{$v_{\rm Earth}$}}
\newcommand{\VelocityEarthValue}{\SI{232}{\km\per\s}}
\newcommand{\VelocityEscapeSymbol}{\mbox{$v_{\rm escape}$}}
\newcommand{\VelocityEscapeValue}{\SI{544}{\km\per\s}}
\newcommand{\WIMPMassSymbol}{\mbox{m_\chi}}
\newcommand{\WIMPMassLowLimit}{\SI{50}{\GeV\per\square\c}}
\newcommand{\WIMPMassHundredGev}{\SI{100}{\GeV\per\square\c}}
\newcommand{\WIMPMassOneTev}{\SI{1}{\TeV\per\square\c}}
\newcommand{\WIMPMassTenTev}{\SI{10}{\TeV\per\square\c}}
\newcommand{\LArNormalTemperature}{\SI{87}{\kelvin}}
\newcommand{\LINNormalTemperature}{\SI{77}{\kelvin}}
\newcommand{\LXeNormalTemperature}{\SI{165}{\kelvin}}
\newcommand{\RoomTemperature}{\SI{300}{\kelvin}}
\newcommand{\ElectronMass}{\SI{511}{\keV\per\square\c}}
\newcommand{\LHCCenterOfMassEnergy}{\SI{13}{TeV}}
\newcommand{\LHCDirectSearchesWIMPMassTurnoverThreshold}{\text{a few hundred~\si{GeV\per\square\c}}}
\newcommand{\BackgroundFreeRequirement}{\SI{<0.1}{\ev}}
\newcommand{\ZeroBackgroundNinetyPerCentCLEventsLimit}{\SI{2.3}{\ev}}
\newcommand{\NinetyPerCentCL}{\mbox{\SI{90}{\percent}~C.L.}}
\newcommand{\ArScintillationYield}{\SI{4E4}{\ph\per\MeV}}
\newcommand{\ArWaveLength}{\SI{128}{\nano\meter}}
\newcommand{\TPBWaveLength}{\SI{420}{\nano\meter}}
\newcommand{\XeWaveLength}{\SI{172}{\nano\meter}}
\newcommand{\NR}{\mbox{NR}}
\newcommand{\NRs}{\mbox{NRs}}
\newcommand{\ER}{\mbox{ER}}
\newcommand{\ERs}{\mbox{ERs}}
\newcommand{\bg}{\mbox{$\beta/\gamma$}}
\newcommand{\bgs}{\mbox{$\beta/\gamma$'s}}
\newcommand{\gr}{\mbox{$\gamma$-ray}}
\newcommand{\grs}{\mbox{$\gamma$-rays}}
\newcommand{\bta}{\mbox{$\beta$}}
\newcommand{\btas}{\mbox{$\beta$'s}}
\newcommand{\LEff}{\mbox{$\mathcal{L}_{\rm eff}$}}
\newcommand{\SOne}{\mbox{S1}}
\newcommand{\STwo}{\mbox{S2}}
\newcommand{\STwoSoneRatio}{\mbox{S2/S1}}
\newcommand{\SThree}{\mbox{S3}}
\newcommand{\GWL}{\mbox{GWL}}
\newcommand{\LWL}{\mbox{LWL}}
\newcommand{\mbb}{\mbox{$m_{\beta\beta}$}}
\newcommand{\qbb}{\mbox{$Q_{\beta\beta}$}}
\newcommand{\bb}{\mbox{$\beta$$\beta$}}
\newcommand{\obb}{\mbox{0$\nu$$\beta$$\beta$}}
\newcommand{\toh}{\mbox{T$^{0\nu}_{1/2}$}}
\newcommand{\tth}{\mbox{T$^{2\nu}_{1/2}$}}
\newcommand{\aSe}{\mbox{a-\ce{Se}}}

\usepackage{fancyhdr}
\usepackage{enumitem}
\usepackage{graphicx}
\usepackage{graphicx}
\usepackage{wrapfig}
\usepackage[title]{appendix}
\usepackage{float}
%\linespread{0.9}
\fancyhf{}
\renewcommand{\headrulewidth}{0pt}
\rfoot{\thepage}
\pagestyle{fancy}
%\linespread{0}
\renewcommand{\thepage}{}
\renewcommand{\thepage}{\arabic{page}}
\renewcommand\thesection{\arabic{section}}
\renewcommand\thesubsection{\thesection.\arabic{subsection}}
\renewcommand\thesubsubsection{\thesubsection.\arabic{subsubsection}}
\parskip = 6pt %changes spacing between paragraphs
%\nolinenumbers
\makeatletter
\def\p@subsection{}
\makeatother
\makeatletter
\def\p@subsubsection{}
\makeatother
%---
\begin{document}
%---
\title{SORA: Stratospheric Organism and Radiation Analyzer}

\begin{abstract}
\begin{center}
{\bf Abstract}
\end{center}
The SORA payload sampled for the existence of microorganisms and bacterial spores in the upper atmosphere. The payload analyzed different aspects of the surrounding environment such as radiation exposure, temperature, pressure and humidity. The payload had three main scientific objectives. First, design and build a novel system that will isolate surrounding air and sample for cells. Second, onboard sensors analyze exposure to solar and cosmic radiation that microorganisms may encounter. Finally, monitor the environmental conditions such as temperature, pressure, and humidity. Furthermore, the design employed additive manufacturing and hobby electronics in its construction to provide an accessible basis for future missions and explore the bounds of the technology available.  SORA successfully isolated organisms and data was received on November 28th, 2017.  Onboard sensors gathered a wealth of information regarding surrounding radiation, with a peak dosage rate of about \SI{0.07}{\micro\Gray\per\minute} at ascent and about \SI{0.05}{\micro\Gray\per\minute} during float. Finally, SORA monitored the environment for the duration of the flight, successfully testing the prototype flight computer, while keeping the power consumption below \SI{1.5}{\ampere} at \SI{30}{\volt}.
\newpage %Breaks page for the Table of Contents.
\end{abstract}
\newcommand{\Houston}{Department of Physics, University of Houston, Houston, TX 77204, USA}
%--- Add other authors in the order they should appear

\author{S.~A.~Garcia~Morelos}\affiliation{\Houston}
\author{F.~Brooks}\affiliation{\Houston}
\author{S.~Oliver}\affiliation{\Houston}
\author{A.~Walker}\affiliation{\Houston}
\author{K.~D.~Portillo}\affiliation{\Houston}
\author{R.~B.~Masek}\affiliation{\Houston}
\author{S.~George}\affiliation{\Houston}
\author{D.~Pattison}\affiliation{\Houston}
\author{A.~L.~Renshaw}\affiliation{\Houston}





\setlength{\parindent}{1em}
\setdefaultleftmargin{1em}{1em}{}{}{}{}
%---
\setcounter{page}{0}\thispagestyle{empty}
%---
\maketitle
\onecolumngrid
\setcounter{tocdepth}{2}
\setcounter{page}{0}\thispagestyle{empty}
\tableofcontents
\setcounter{page}{0}\thispagestyle{empty}
\newpage
%---
\onecolumngrid %changed from "twocolumngrid"

%Section: Design
%\input{Design.tex} %This needs to be removed as a section and only the one below this (SORA Hardware Description) will be left in place.

%Section: Introduction
\section{Mission Statement and Objectives}
\label{sec:Introduction}

Our goal for the HASP 2018 payload is to further build upon the first SORA~\cite{SORA} flight.  In order to confirm our findings from SORA 2017, we need to again collect extremophile bacteria that reside in the upper atmosphere at approximately 36 to 41 kilometers.  A MiniPIX particle detector will be flown onboard to further study the effects of ionizing radiation on these living organisms and other various sensors will gather data pertaining to the environmental conditions in which these extremophiles reside. Due to the limited amount of data for the 20 to 40 kilometer altitude range, we have generated questions, hypotheses and objectives, based on past HASP payloads and other high altitude collection flights, that have thus far remained unanswered or have little corroborating data.

%\begin{figure}[h!]
%  \begin{center}
%    \begin{minipage}[c]{0.45\linewidth}
%      \includegraphics[width=\textwidth]{./Figures/sora_takeoff.jpg}
%      \caption{HASP platform at launch with the SORA payload onboard.}
%      \label{fig:takeoff}
%    \end{minipage}
%    \hfill
%    \begin{minipage}[c]{0.49\linewidth}
%      \includegraphics[width=\textwidth]{./Figures/sora_flight.jpg}
%      \caption{HASP platform at float, the SORA payload is the only gold foil covered payload onboard.}
%      \label{fig:float}
%    \end{minipage}
%  \end{center}
%\end{figure}

The main goals for SORA are to collect extremophile organisms that reside in the upper atmosphere, study the effects of surrounding radiation on these organisms in the stratosphere and gather data pertaining to the environmental conditions in which these organisms reside~\cite{SORA}.  More specifically, SORA has two sets of main objectives, along with four additional objectives.

{\bf Primary Scientific Objectives:}
	\begin{enumerate}
	\item Attempt to capture microorganisms in the upper atmosphere at approximately \SIrange{30}{41}{\kilo\meter} of altitude using multiple methods. 
	\item Culture samples and compare the collection medium.
	\item Study the cosmic and terrestrial radiation in which these extremophiles may reside.
	\end{enumerate}
	
{\bf Secondary Scientific Objectives:}
	\begin{enumerate}
	\item Test RESU further and develop a more power-efficient flight control system.
	\item Determine the polar angle of hits on the detector and be able to compare them to payload orientation information.
	\cmt{Andrew W.} {I'm beginning to doubt the usefulness of this aspect as according to Stuart's research on the ISS, measurements of GCRs are isotropic.(Meaning we would probably see very little correleation in terms of polar angle even if it was corrected for payload orientation)}
	\item Further testing of the astrobiology hardware in flight and the methodology for collection of microbes in extreme environments at high-altitude.
	\item Improve pre-and post-flight decontamination procedures.
	\end{enumerate}

\cmt{Andrew W.}{Maybe in terms of science we should focus more on the biological effect of radiation.}

{\bf Engineering Objectives}
	\begin{enumerate}
	\item Implement a variable shutter time for the MiniPIX based on the flux of particles incident on the detector.
	\item Analyze MiniPIX data in real time and downlink relevant radiation statistics.
	\item Implement a redundant data storage mechanism.
	\item Test an improved enclosure against impacts and harsh environments.
	\item Improve astrobiology collection mechanism.
	\end{enumerate}

These goals and objectives are based on the following scientific questions: After confirming that microorganisms are present in the upper atmosphere in our last mission~\ref{SORA}, what extremophiles are present in the upper atmosphere at altitudes of 36 to 41 km?  If extremophiles are captured, can we culture the microorganisms?  What methods are more effective at capturing bacteria for culturing? Finally, with a deeper understanding of the MiniPIX after our first mission, can we collect more data to study cosmic radiation that microoganisms and spores are exposed to on a daily basis? Specifically, can we obtain useful information about the biological effectiveness of this radiation on bacteria through parameters such as linear energy transfer and dose equivalent. \cmt{Andrew R.}{It would be good to revamp or expand on these questions based on the slightly new objectives.}
\cmt{Sam}{I made some changes, I think this would be more fitting.  Andrew W. had a few more ideas for the MiniPIX that we could add}
\cmt{Andrew W.} {Added a more specific MiniPIX scientific goal let me know what you think}

%\begin{itemize}
%	\item Are extremophiles present in the upper atmosphere at altitudes of 36 to 41 km?
%	\item Will the container design protect all components of the data collection system and allow for accurate results?
%	\item Will the Clean Box design prevent sample contamination?
%	\item What are the effects of cosmic and UV radiation to organisms and bacterial spores?
%\end{itemize}

\subsection{Hypothesis and Objectives}
\label{subsec:Hypothesis-Objectives}
\begin{enumerate}
\item Based on the collection results from previous missions such as SORA~\cite{SORA}, we predict the concentration of cells at an altitude of 36 km will be less than 500 cells per liter \citep{LSU}.
	\begin{enumerate}
	\item Objective: Sample a minimum volumetric amount of air at target altitude for the duration of the float phase (approximately 15 to 18 hours).
	%\item Status: This objective was completed in flight, with successful operation of the sampling pump for the duration of the flight, operating at a confirmed current which allowed for air flow through the sample cell for the entire float portion of the mission.  As of November 28th, 2017 the RNA sequencing results from the lab analysis arrived with positive results.  The full interpretation of the results is currently underway as seen in Section~\ref{sec:Astrobiology-Results}.
	\end{enumerate}
\item Based on control samples and testing before flight, we can compare our final flight results to previous applications.
	\begin{enumerate}
	\item Objective: Quantify and characterize any contamination with our laboratory and payload disinfection procedures.
	\item Objective: Minimize the amount of external contamination before flight with thorough decontamination procedures.
	%\item Status: Study is currently underway based on the external lab analysis results which returned RNA sequencing for the collected sample, as well as the control sample.
	\end{enumerate}
%\item Using SolidWorks 3D CAD design software~\cite{Solidworks} and COMSOL Multiphysics Simulation software~\cite{COMSOL}, we can predict what kind of conditions and forces our container will experience.
	%\begin{enumerate}
	%\item Objective: Using results from simulations and real world tests, adapt a structure and a container to ensure the final results are reliable.
	%\item Status: Incomplete - no simulation results.  Main issue was not having enough time to complete in-depth simulations and actual experimental testing proved more reliable.
	%\end{enumerate}
\item Based on measured results of dosage rates, the higher exposure to radiation may change the organism's cellular make-up.
	\begin{enumerate}
	\item Objective: Quantify the intensity and exposure cosmic radiation over the duration of the flight.
	\item Objective: After capturing samples, analyze data and compare biological effects to similar genotypes found on Earth's surface.
	%\item Status: After consultation with biologists at University of Houston (UH), we chose to send the sample for sequencing to see if we captured anything, thus making it hard to compare directly the differences between anything found on Earth and in the atmosphere.  Another mission is necessary to fully complete this objective, but positive results from this mission show the possibilities for a follow-up mission.
	\end{enumerate}
\end{enumerate}

\vspace*{-0.5cm}

%Section: IntroAstrobiology
\subsection{Astrobiology}
\label{sec:Astrobiology-Background}

Extremophiles thrive in physically and/or chemically extreme conditions, which are detrimental to most of life on Earth as we know it. These organisms and microbes have been found everywhere, from deep underwater volcano vents to buried ice lakes in Antarctica \cite{Extremophiles}. Conditions at altitudes of \SI{30}{\kilo\meter} to \SI{40}{\kilo\meter} are extreme in temperature, pressure and radiation. As shown in Table~\ref{tab:astrobiotable}, fungi and bacterial spores have previously been found in the stratosphere. Arguably, each successful collection expedition of at least \SI{30}{\kilo\meter} into the upper atmosphere provides information that could be useful in determining what life forms can exist outside of Earth's biosphere. 
	
%\begin{figure}[H]
%\centering
%\includegraphics[width=1\textwidth]{./Figures/AstroChart.PNG}
%\caption{}
%\label{fig:AstroHist}
%\end{figure} 

Today, the most common altitude for bacterial collection in the atmosphere occurs in the range of approximately \SIrange{10}{20}{\kilo\meter} above Earth's surface; very little data exists on microbiological samples captured in the stratosphere. Conditions at altitudes of \SIrange{30}{40}{\kilo\meter} are extreme in temperature, pressure and radiation. 



Our experiment is an attempt to further develop new and innovative techniques for capturing microorganisms in the upper atmosphere, as demonstrated during the LSU HASP 2011, 2012, 2013 flights~\cite{LSU} and by D.R. Canales~\cite{canales}. We used the KNF N84-4 commercial gas-sampling diaphragm vacuum pump to sample the air at approximately \SI{33}{\kilo\meter} above Earth's surface. The samples we collected are an important part to expanding our understanding of Earth's biosphere and further studies could provide more insight on how life can be distributed on Earth, and ultimately, through outer-space.

\begin{table}[!ht]
\centering
\caption{History of Microbiological Sampling of the stratosphere~\cite{SORA}.} 
\label{tab:AstroHist} 
\bigskip
\begin{tabular}{|c|c|c|p{6cm}|c|}
\hline
\multicolumn{1}{|c|}{\bfseries Date} & \multicolumn{1}{c|}{\bfseries Altitude (km)} &  \multicolumn{1}{c|}{\bfseries Sample Method} & \multicolumn{1}{p{6cm}|}{\bfseries Biology Measured} & \multicolumn{1}{c|}{\bfseries Volume} \\
\hline
    1936	& 11 - 12 	& Balloon			 			& \minitab{l}{5 Bacillus sp., 1 Penicillium sp.,}{1 Macrosporium sp., 2 Aspergillus sp.} 			& $Unknown$ \\ \hline
    1978	& 48 - 77 	&Meteorological Rocket	 		& Mycobacterium sp., Mircococcus sp.					       							& $Unknown$ 	\\ \hline
    2003	& 30 - 41	& Balloon, liquid neon cryopump	& \minitab{l}{Isolated S. pastuerii, B. simplex,}{the fungus, Egnydontium album}       				& $57$	\\ \hline    
    2004	& 20	 	&Airplane, Impactor Surfaces 	 	& Bacillus luciferins, Bacillus sphaericus			       									& $Unknown$ 	\\ \hline
    2006	& 19 - 41	& Balloon, Liquid Neon Cryopump 	& \minitab{l}{7 cells L-1 (counting clumps), Bacillus sp.,}{Staphylococcus sp., Engyodontium sp.}	& $19-81$ \\ \hline
    2007	& 20	       	& Airplane, Impactor Surfaces 		& \minitab{l}{Micrococci, Microbacteria,}{Staphylococcus sp., Brevibacterium sp.}    				& $Unknown$ \\ \hline
    2010	& 20	       	& Airplane, Impactor Surfaces 		& Isolated Bacillus sp.							     								& $Unknown$\\ \hline
\end{tabular}
\label{tab:astrobiotable}
\medskip
\end{table}

%Section: IntroRadiation
\subsection{Radiation Introduction}
\label{sec: Radiation Background}

\subsubsection{Cosmic Radiation}
The study of the biological effects of cosmic radiation in space or near space environments is important for human space travel. Long term space travel necessarily requires humans and other biological specimens to be exposed to high levels of radiation for extended periods of time, so understanding amount of radiation exposure experienced from GCRs has important applications for flights to Mars and beyond. Also, the understanding of  cosmic rays specifically in Earth's atmosphere also has important applications to commercial airplane flights as they generally operate in a layer of the  atmosphere with far higher levels of radiation exposure than at the surface of Earth. 

Our goals for the radiation portion of our payload are two fold, to measure radiation levels at various layers in the atmosphere to determine its possible effect on microorganisms, and to test the operation of the MiniPIX particle detector in the extreme environment of the stratosphere for possible use in future missions. We ultimately seek to further previous research on utilizing Medipix/Timepix devices for measuring GCRs on stratospheric balloon flights~\cite{bexus}.
%
%	The biological effectiveness of
%Galactic Cosmic Rays (GCR) in space has been studied on
%microorganisms through the use of computer modeled radiobiological
%systems using Monte Carlo simulations. With an increase in intensity
%of GCR's, the number and magnitude of damage in cells increase 5. Such
%damage occuring from GCRs can effect the biological system.
%
%In the atmosphere ionizing radiation does not always increase with altitude. As
%primary GCRs such as high energy protons, alpha particles, and 
%heavy ions collide with atoms in the atmosphere they begin to shatter into secondary
%particles such as neutrons, pions, electrons and muons which causes a peak in 
%ionizing radiation at around \SI{18}{\kilo\meter}. This peak is known 
%as the Regener-Pfotzer Maximum\cite{regener} and shows that with increasing atmospheric density
% ionizing radiation increases until peaking high in the stratosphere and then decreases rapidly as you 
%reach the surface of the earth.



%Expand upon Regener-Pfotzer Maximum. Refer to other papers to validate the range. 
%The intensity of GCR peaks within a range of about
%\SI{18}{\kilo\meter} to about \SI{22}{\kilo\meter} []. This range,where the production of ionizing radiation reaches its
%peak is known as the Regener-Pfotzer Maximum\cite{regener} . The Regener-Pfotzer Maxiumum is unique and is dependent on
%location and time of the year, as it is determined by a number of
%factors, which include but are not limited to the strength of earth's
%electromagnetic field, atmospheric composition (specifically ozone
%content(?){\bf The intensity of the cosmic ray flux and the secondary
%  environment vary inversely with the solar cycle due to the
%  interaction of the earths electromagnetic field. In addition, the
%  sporadic solar events that occur in short busts can increase the
%  primary particle flux periodically (hours to days) can in fact
%  enhance the atmospheric radiation several orders of magnitude in
%  scale.}), the sun's relative position, and maximum solar elevation
%[]. The combination of these affects results in a variability in the
%location of the maximum as well as the existene of this maximum as
%ooposed to an ever-increasing intensity.

\subsubsection{Solar Radiation}
The ultraviolet (UV) spectrum is composed of UVC (\SIrange{200}{280}{\nano\meter}) with only \SI{0.5}{\percent} of the entire solar spectrum, 1.5\% of
UVB (\SIrange{280}{315}{\nano\meter}) and UVA (\SIrange{315}{400}{\nano\meter}), which contributes to \SI{6.3}{\percent}~\cite{uv_irradiance}. UVB and UVC are the main contributors in highly lethal solar radiation to microorganisms\cite{UVonDNA}. DNA is prone to high absorption levels at those wavelengths, often causing inactivation and mutation.  Therefore, understanding the exposure of microbes in to UV radiation is quite important.



%Section: SORA Hardware Description
\section{SORA Payload Description}
\label{sec:Hardware}
This is the description of the payload.

%\begin{figure}[!htb]
%\begin{center}
%\includegraphics[width=0.5\textwidth]{./Figures/Test.jpg}
%\caption{This is a test figure.
%}
%\label{fig:Test} 
%\end{center}
%\end{figure}
\subsection{RESU Design}
\label{sec:RESU-Design}

For this mission, we will again use RESU (Real-Time Environmental Sensing Unit), our flight computer that can manage flight operations as well as send and receive serial uplink and downlink packets. RESU is composed of two components: a Raspberry Pi 3 (RP3) and an Arduino Mega (Arduino) which interfaced over serial USB.  The RP3 and Arduino both are hobby electronic computers, but the RP3 is geared towards recording and processing while the Arduino is more efficient at sensor integration.  The Arduino's primary purpose is to collect data from various sensors that monitor environmental conditions. It also can handle telemetry, periodically downlinking packets and accepting uplink commands received from the ground control. The RP3 records data from the Arduino's serial port and saves data frames every six seconds from the MiniPIX.  Table~\ref{tab:Sensors} lists the various sensors that will be utilized during flight.

\begin{table}[h!]
\centering
\caption{Table of sensors that compose RESU}
\label{tab:Sensors}
\bigskip
\begin{tabular}{|c|c|c|c|}
\hline
\multicolumn{1}{|c|}{\bfseries Sensor} & {\bfseries Quantity} & {\bfseries Platform} & {\bfseries Purpose} \\
\hline
    Temperature Sensors (TMP 36)	& 3 & Arduino  		& Record temperature measurements  \\ \hline
    Pressure        				& 1 & Arduino 		& Record pressure measurements \\ \hline
    BNO 6055       					& 1 & Arduino 		& Record IMU Data in 9 degrees of freedom \\ \hline    
    Real Time Clock 				& 2 & Arduino/RP3 	& Record temperature compensated timestamps in CT \\\hline
    Humidity        				& 1 & Arduino 		& Record atmospheric humidity levels \\ \hline
    GPS     						& 1 & Arduino 		& Determine latitude, longitude, altitude and direction \\ \hline
    MiniPIX         				& 1 & RP3     		& Cosmic ray detector \\ \hline
\end{tabular}
\end{table}

\subsubsection{Electronics Design}
Since much of the space inside of our payload will be taken up by the pumps and clean box from the astrobiology systems we had to design our electronics to be relatively compact.  It was decided that we would only use one RP3 to both interface with MiniPIX and store sensor data from the Arduino.  Also, in order to reduce the space required for the interface between the Arduino and all of the payload's sensors, we used two layers of proto shields to more effectively utilize vertical space.  The RTC, pressure, humidity, and inertial measurement sensors are all mounted directly on the first shield while the temperature and GPS sensors are mounted on to the top most shield. 

\subsubsection{Telemetry}
RESU was designed to handle every component of during a mission.  It is in charge of all  telemetry to and from the HASP systems.  Downlinked packets will provide timestamped readings from all the payload's sensors which help us in analyzing the current state of our payload.  RESU also was programmed to receive four uplink commands: heaters on/off and pumps on/off.  These commands can be sent at our discretion which allow us to manage our collection systems at any given point during the flight based on environmental and component temperatures. 

\subsection{Astrobiology System Design}
\label{sec:Astrobiology Design}

The collection assembly will be designed as a multi-compartment structure, with various collection and control containers, two pumps, two pump heaters, and two solenoids.  The control containers will be connected to a solenoid that will remain closed until post-sanitation procedures are performed. The sample collection containers will be connected to a vacuum pump located outside of the clean box structure.  One of the chambers will be a non-liquid collection container. Heaters will be attached to the pumps to aid during a cold start. Once float altitude is reached, the solenoids connected to the sample collection containers will be opened and the pumps will be powered on; allowing air to flow to the collection containers.The containers will hold a range of amounts and concentrations, \SIrange{30}{60}{\milli\liter} of \SIrange{15}{30}{\%} sterile glycerol solutions. A 316 Stainless Steel \SI{1/4}{\inch} NPT Vent to Atmosphere Vitron Seal Valve will be embedded in each of the compartments, to accommodate for the pressure changes that occurs with the variations in altitude over the course of a flight~\cite{valve}, as well as operate as the sample exhaust.  The left side of Figure~\ref{fig:pump} displays the collection assembly with the openings for the sample and exhaust tubing, while the right side of Figure~\ref{fig:pump} shows the 3D rendering of the sampling pump.  

\begin{figure}[!h] 
\begin{center}
\includegraphics[scale=.4]{./Figures/CB.PDF}
\includegraphics[scale=.8]{./Figures/Pump.pdf}
\caption{{\bf Left:} Cross-section view of the clean box with experimental and control containers. {\bf Right:} KNF N84-4 commercial gas-sampling diaphragm vacuum pump.}
\label{fig:pump}
\end{center}
\end{figure} 

\subsection{Radiation Monitoring System Design}
\label{sec:Radiation Design}
\subsubsection{MiniPIX Detector}
%cite{advacam} is getting an error, leave blank and add comment
The MiniPIX detector, shown in Figure~\ref{fig:minipix} is a silicon-based hybrid pixel detector built by ADVACAM~\cite{advacam} that utilizes technology developed by the Medipix2 collaboration at CERN~\cite{medipix}. The sensor is composed of a pixellated silicon sensor integrated with a single Timepix readout chip (256 x 256) pixels with a pitch of \SI{55}{\micro\meter}, the layers of the detector are shown in Figure~\ref{fig:minipixlayers}. The sensor is \SI{500}{\micro\meter} thick and uses a USB 2.0 interface with a readout rate up to 30 frames per second. Each pixel can be programmed to work in one of three modes: Single particle counting, Time-over-Threshold (TOT), or Time-of-Arrival (TOA). This device offers a variety of applications including X-ray and neutron imaging as well as particle identification by characterizing each particle due to their charge, energy, and direction.

\begin{figure}[h!]
  \begin{minipage}[c]{0.40\linewidth}
    \includegraphics[width=\linewidth]{Figures/minipix_detector.png}
    \caption{Picture of a MiniPIX particle detector~\cite{advacam}.} %make sure to put figure names underneath the pictures
    \label{fig:minipix}
  \end{minipage}
  \hfill
  \begin{minipage}[c]{0.45\linewidth}
    \includegraphics[width=\linewidth]{Figures/Silicon_sensor.png}
    \caption{Hybrid pixel detector silicon sensor~\cite{silicon_sensor}.} %make sure to put figure names underneath the pictures
    \label{fig:minipixlayers}
  \end{minipage}
\end{figure}

The MiniPIX registers ionized particles when the active material in the detector is transformed into a charge (the excitement of electrons-hole pairs in the semiconductor). When these charges are collected, the Si bulk is then depleted by an applied bias voltage, which occurs when the electronics reads-out electron-hole pairs. If these pairs are above the threshold when collected by the pixel electronics, the count increases. The projection of the deposited charge is measured and the total energy deposited can be determined from the back-plane pulse amplitude. 

\begin{figure}[!h]
  \begin{minipage}[c]{0.49\linewidth}
    \includegraphics[scale=1, width=.5\textwidth]{Figures/Minipix_case_assembly.pdf}
    \caption{3D rendering of the MiniPIX case and heat sink assembly.} %make sure to put figure names underneath the pictures
    \label{fig:case_assem}
  \end{minipage}
  \hfill
  \begin{minipage}[c]{0.49\linewidth}
    \centering
    \includegraphics[scale=1, width=.5\textwidth]{Figures/minipix_mounted.png}
    \caption{Picture of the MiniPIX mounted in the case and heat sink assembly.} %make sure to put figure names underneath the pictures
    \label{fig:case_assem_pic}
  \end{minipage}
\end{figure}

When a particle is incident on the sensor, a particle track is produced along the path length through the sensor area. The path of a single particle through the sensor is called the particle track.  Each particle track may be identified as a ``cluster'' or continuous area of neighboring pixels in a given frame. Each individual cluster can be differentiated through statistical analysis to describe the shape and energy deposition per frame. Thus, allowing us to distinguish between different types of radiation which by organizing each cluster into specific morphological categories. The feature parameters of this detector provides detailed information about total energy deposited by particles, and if only one particle is traced in the detector this tells us that slow heavy particles ionize more than fast moving ones. 

The primary purpose of the MiniPIX was to detect four specific types of radiation: alpha $(\alpha)$, electron $({e^-})$, gamma $({\gamma})$, and muon $({\mu})$. By comparing the results of the flight to results obtained from simulations, an estimate regarding the percent composition of the detector hits can be made.

The MiniPIX case and heat sink assembly shown in Figure~\ref{fig:case_assem} was designed to protect the device from any moisture in the atmosphere. The case and lid were 3D printed in ABS plastic and the heat sink, composed of two large sheets and a small block of aluminum metal were all mounted to the roof of the payload. Thermal paste was applied between the contacts of the device and all aluminum pieces to allow for optimal thermal conduction and radiation of heat away from the device. A picture of the MiniPIX device inside the case and heat sink assembly is shown in Figure~\ref{fig:case_assem_pic}.

\subsubsection{UV Photodiode Array}
The UV photodiode array is composed of three SiC UV Photodiodes~\cite{JIC 139}. The characteristics of the photodiodes allow measurements within the spectral range between \SI{210}{\nano\meter} - \SI{390}{\nano\meter}. The maximum irradiance allowed for measurement is \SI{189}{\milli\watt\per\meter\squared}. A layer of two UV neutral density (ND) filters~\cite{Thor Labs} with optical densities (OD) 1.3 and 2.0 were positioned above each diode to reduce the transmission of UV light entering the sensor window. 3D renderings of the apparatus and filter are shown in Figures~\ref{fig:UVArray} and~\ref{fig:NDFilter}. This setup ensured that saturation of the diodes would not occur.

\begin{figure}[!h]
  \begin{minipage}[c]{0.49\linewidth}
    \includegraphics[scale=1, width=.5\textwidth]{Figures/uvapparatus.JPG}
    \caption{3D rendering of the UV photodiode array.} %make sure to put figure names underneath the pictures
    \label{fig:UVArray}
  \end{minipage}
  \hfill
  \begin{minipage}[c]{0.49\linewidth}
    \includegraphics[scale=1, width=.5\textwidth]{Figures/uv_filter.JPG}
    \caption{3D rendering of an unmounted UV ND filter.} %make sure to put figure names underneath the pictures
    \label{fig:NDFilter}
  \end{minipage}
\end{figure}

The Arduino from RESU was used to supply a constant \SI{5.0}{\volt} signal to each detector. When a UV photon is incident on the sensor, a photocurrent is induced in the circuit and a voltage pulse is returned to the Arduino and stored on the RP3. The sensor output current can be calculated from the measured voltage and the spectral irradiance can be calculated based on the characteristics of the detector. SORA used this apparatus to quantify the spectral irradiance of solar UV light in the stratosphere.


%Section: Methods
%\newpage
\section{Methods}
\label{sec:Methods}

%\begin{figure}[!htb]
%\begin{center}
%\includegraphics[width=0.5\textwidth]{./Figures/Test.jpg}
%\caption{This is a test figure.}
%\label{fig:Test} 
%\end{center}
%\end{figure}
%\clearpage
\subsection{Electronics System Design}
\label{sec:RESU}

\subsubsection{Overview}

The RESUs primary purpose during the flight will be to monitor environmental conditions and control the astrobiology systems. It will monitor temperature of the various subsystems and the humidity and pressure of the environment throughout the flight. All recorded data will continuously be written to an SD card mounted on a shield on top of the Arduino. It will also accept discrete commands from the HASP systems to turn the astrobiology collection system on and off.

\subsubsection{The Sensors}

Our payload will utilize eight thermistors to measure temperature at various points in our payload. The decision to use thermistors was based primarily on the performance of the analog temperature sensors during our 2017 flight, during which several of those sensors failed. Thermistors are able to accurately measure temperature in the range \SIrange{-55}{125}{\celsius} and should therefore be adequate for the conditions in the stratosphere. Pressure will be recorded from two identical digital pressure sensors in order to ensure accuracy and should be able to record accurately in the range \SIrange{0}{14000}{\milli\bara}. Finally, humidity will be measured from a basic analog humidity sensor. All sensor data will be UTC timestamped via the onboard Real Time Clock and recorded to the SD card.

 %UVC radiation was measured by three identical sensors.  They supported light ranging from \SIrange{210}{380}{\watt\per\square\meter} and were located outside the roof of the payload.  These sensors were analog so their readings were voltages induced by incident rays on the surface of each sensor. To stabilize the voltage readings from each sensor, a \SI{3300}{\micro\farad} capacitor was placed across the ground and analog out pins of the sensor.  Data was collected every \SIrange{3}{4}{\second}.


% \subsubsection{Powering It All Up}

% In order to stay within the power constraints, a robust power supply will need to  handle all the components of the payload.  The power supply we will be using is the PPM-DC-ATX-P by WinSystems INC.  It offers the desired number of \SI{+5}{\volt} and \SI{+12}{\volt} outputs needed to power the payload's electronics.  This power supply could effectively take \SI{+30}{\volt} and step it down to two \SI{+12}{\volt} and two \SI{+5 }{\volt} outputs.  One of the \SI{+12}{\volt} outputs goes to the Arduino since it can step down to the appropriate voltages internally while the other goes to a PWM motor for the solenoid.  One of the \SI{+5 }{\volt} outputs powers two analog sensors that will be sent to HASP through the EDAC connection (more on that in the next sections).  The remaining \SI{+5 }{\volt} output is converted to a USB power cable for the RP3.  The power supply also has four ground outputs that will be used by each respective component. 


%Listing~\ref{Downlinks} is a sample downlink data packet received during our previous flight.

%\lstset{basicstyle=\small, numbers=left, xleftmargin=2em, frame=tb, label = Downlinks, framexleftmargin=1.5em}
%\begin{lstlisting}[caption = Sample of downlinked data packets ID: 15667 - 15670 from SORA 2017~\cite{SORA}.]
%...
%begin_packet
%15667,-21.05,-0.33,35.62,-1.00,0.12,0.25,-0.71,0.63,9.30,-30.25,-26.69,-60.88
%3.14,2.86,0.00,0.00,0.60,9:29:39 9/4/17
%end_packet
%begin_packet
%15668,-20.95,-0.54,35.62,0.75,0.06,-0.69,0.25,-2.00,9.67,-28.37,-28.06,-60.06
%3.80,3.33,0.00,0.00,0.60,9:29:42 9/4/17
%end_packet
%begin_packet
%15669,-21.15,-0.54,35.62,0.81,-0.06,-0.56,1.22,-2.13,9.86,-29.56,-27.75,-57.88
%1.86,2.86,0.01,0.02,0.61,9:29:45 9/4/17
%end_packet
%begin_packet
%15670,-21.15,-0.54,35.62,-1.00,0.00,0.12,-0.55,0.73,9.37,-27.37,-30.00,-58.75
%1.86,3.96,0.00,0.00,0.61,9:29:48 9/4/17
%end_packet
%...
%\end{lstlisting}
%\medskip
%
%Each packet of data will be delimited by keywords \verb|begin_packet| and \verb|end_packet| so that parsing each file be easier.  During our previous flight, these data packets were crucial status updates to the state of our payload.  We will once again use them for the same purpose to keep a close eye on our payload.
%
%Data packets will contain information about the readings from each sensor.  The first integer represents the ID of the packet.  Following the packet ID are the three temperature readings, gyroscope $x, y, z$ values, accelerometer $x, y, z$ values, magnetometer $x, y, z$ values, pressure readings, humidity readings, UV \#1, \#2, \#3 voltage readings, and finally the timestamp in $HH$:$MM$:$SS$ $MM$/$DD$/$YY$ at which the packet was written.  
%
%In addition to downlinking sensor data, we also want to downlink serial uplink commands as shown in listing \ref{Uplinks}.  
%
%\lstset{basicstyle=\small, numbers=left, frame=tb, linewidth=11.5cm, xleftmargin=.4\textwidth, label = Uplinks}
%\begin{lstlisting}[caption = Sample of received uplink commands in downlinked packets in SORA 2017~\cite{SORA}]
%...
%1
%2
%71
%FFFFFFFF
%3
%D
%A
%Received Command: 71
%...
%\end{lstlisting}
%\medskip
%
%Each line within a received uplink command represents the string of bytes that will be read by our payload.  The last line of the listing shows what command will be parsed and processed by RESU.  Table \ref{tab:All-Commands} shows all the possible commands that we are expecting to process. 

\subsection{Astrobiology Methods}
\label{sec:Astrobiology Methods}

\subsubsection{Vacuum Chamber Testing}


The pumps, along with the temperature, humidity and pressure sensors and tubing will undergo extensive thermal vacuum testing in the range of \SIrange{-3}{25}{\celsius}, with a pressure range of \SIrange{0}{10}{\milli\bar}. The vacuum chamber tests will run for approximately \SIrange{8}{15}{\hour}, to replicate conditions from the previous flights. In the past~\cite{SORA}, the pumps were subjected to a temperature range of \SIrange{-30}{50}{\celsius} during integration and ran again for \SI{8}{\hour} afterwards and prior to flight. The pumps have proven to function in extremem environments without an issues. 


\subsubsection{Pre-Flight Preparation}


The clean box, collection containers and tubing will be autoclaved. All tools used in the assembly of the clean box will either be autoclaved or soaked in a \SI{70}{\percent} ethanol solution inside of a clean room. Each person who enters the clean room will be garbed in a lab coat, goggles, hair net and latex gloves after thoroughly washing their hands in a \SI{70}{\percent} ethanol solution. 

%We need to reword this to fit what Dr. Pattison wants us to do now that we have her onboard

%The \SI{15}{\percent} glycerol solution will poured into each container, the lid was sealed with silicone gasket maker and the tubing was inserted and gasket sealed into the container lids. Each lid had two holes, one that led to the inside of the clean box to allow for pressure to be released from inside the container and outgassed through the valve embedded in the box, while the other hole passed through the clean box lid to allow the tubing to connect to the pump or solenoid only in the case of the control tubing. The lid to the clean box was sealed with silicone gasket maker, the box was then mounted onto the payload and the tube from the control container was clamped to the dedicated control solenoid, while the sample collecting tube was passed through the other solenoid and connected to the pump. A final piece of tubing was connected from the intake valve on the pump to the outside of the payload, after a \SI{70}{\percent} ethanol solution was run through the pump several times. The end of the tube was bent, and zip tied. The payload was closed and remained in the clean room until it was ready for transport to Fort Sumner.  At the launch site, the zip tie was cut and the exposed inner and outer tubing was swabbed with an alcohol pad approximately \SI{4.5}{\hour} prior to launch. 



\subsubsection{Post Flight Procedures}

%Reword this section too to match what we plan to do after we recover it.  Include first that we will put the cleanbox in an iced cooler.

%The payload was successfully retrieved on September~6,~2017. The intact clean box was removed and placed inside of a cooler with ice, transported to The University of Houston and placed in cold storage at \SI{4}{\celsius}. All equipment used in the filtration process was either autoclaved or taken from previously unopened sanitized packaging. The autoclaved, pre-sanitized items and the clean box were washed in a \SI{70}{\percent} ethanol solution before they were placed inside a SterilGARD e3 Class II Biological Safety Cabinet. The Cabinet has a laminar flow air barrier and UV lights built into the ceiling for decontaminating the workspace prior to use. Both the control and sample collection solutions were vacuum filtered through a Fluropore membrane filter (\SI{13}{\milli\meter}; \num{0.22} micron) to collect specimens on the filter surface.  The filters were packaged for shipment to RTL Genomics~\cite{RTL} for 16S ribosomal RNA sequencing. All post flight sanitation and sample and control filtration procedures took place under the supervision of Professor Donna Pattison from the Department of Biology and Biochemistry at The University of Houston.


%\subsubsection{Ribosomal RNA Sequencing and Data Analysis}
%
%
%
%We sent our filtered control and experimental samples to RTL Genomics in Lubbock, Texas for ribosomal RNA sequencing and data analysis. We selected the 926wF  (``AAACTYAAAKGAATTGRCGG'') and 1392R (``ACGGGCGGTGTGTRC***'') primers for the sequencing procedure. These primers are designed to amplify 16S RNA from bacterial, archaeal and eukaryotic ``universal'' samples. The samples were amplified using a two step PCR procedure~\cite{lemon}. Samples were sequenced on an Illumina MiSeq~\cite{illumina} 2x300 flow cell at 10~pM and sequenced at RTL Genomics using standard sequencing procedures~\cite{Microbial rRNA sequencing}.


\subsection{Cosmic Radiation Methods}
\label{sec:Radiation Methods}

\subsubsection{Calibration}   
The appropriate calibration of the MiniPIX detector will be applied at The University of Houston by Dr.~Stuart P.~George, a collaborator within the Medipix Collaboration. The source calibration will be applied using the \SI{60}{\keV} {\ce{^241Am} decay line, \ce{Sn} Fluorescence and \ce{^55Fe} gamma rays. The Timepix hybrid pixel detector consists of \num{65536} silicon p-n diodes, with each containing its own individual processing circuit. The response of each pixel can never be identical, thus a calibration must be performed for each individual pixel. This is a fairly complicated procedure and was our first setback. Dr.~George will calibrate the pixel energy threshold from DAC counts to energy~\cite{stuartthesis}. The threshold will be set at \SI{4}{\keV}, just above the noise level of the detector. The set threshold energy of the Timepix chip determines what energies of particles are allowed to be measured by the detector. Energy measurements in the detector are accounted by measuring the charge collected in each individual pixel. The sensor's bias voltage will be \SI{200}{\volt} to ensure that the sensor was completely depleted. 
  
\subsubsection{Collection Parameters}
% Put the settings for the minipix shutter time, bias voltage etc. here
The device will be configured via the device Python API wrapper provided by ADVACAM. Acquisition parameters will be determined through testing. This time must be chosen as a balance between too many individual frames with little to no data, which would take up a lot of storage space, and individual frames that are so crowded with interactions that they are unreadable due to a large number of crossed tracks. Upon completing an acquisition, the device enters a manufacturer-set dormant state for approximately two seconds, which will be utilized as cool-down period and to record the internal temperature of the device into a .csv file.

\subsubsection{Data Format}
The data format for each frame of data is a plain text array with each value corresponding to the time over threshold value at the corresponding pixel index in the detector. Upon the capture of a new frame the plain text array was appended to the acquisition file stored on the SD card of the RP\num{3} on our payload. In a separate file various metadata is stored for each frame including the detector threshold, acquisition time, acquisition mode etc. The plaintext array format means that each frame of data utilized approximately \SI{132}{\kilo\byte} of memory. In SORA \num{1.0}, a frame was saved roughly every six seconds, over the course of the roughly fourteen hour flight we had a total acquisition size of approximately \SI{1.1}{\giga\byte}. While this was a rather small total collection size, we could have further reduced our memory footprint by only storing pixel indices with non-zero time over threshold values i.e.\ storing data in a sparse matrix format. This is a route we will look into for SORA \num{2.0}.

%\subsubsection{Clustering Algorithm}  	
%\begin{figure}
% 	\begin{center}
% 	\includegraphics[width=0.5\textwidth]{./Figures/crossedtracks.png}
% 	\caption{Frame with two detector hits with crossing paths (bottom-center and shown inside red box).}
% 	\label{fig:crossedtracks}
% 	\end{center}
%\end{figure}
%
%  	Given our raw data frames after flight we needed to perform a clustering algorithm to break up each frame of data into a series of relevant parameters for each individual particle hit. It was decided that given its ease of development we would implement this algorithm in Python utilizing the Numpy scientific computing library~\cite{numpy}. While lacking the speed of C/C++, Python allowed us to quickly develop and debug the algorithm so that we could have more time to analyze the results. However, what we gained in implementation speed we lost in execution time and interactivity. Python proved to be quite slow compared to a compiled language and was rather clumsy in terms of providing interactive plots. It also took many hours each time we wanted to reanalyze our flight data. We could have avoided much of this by either writing the more processing intensive algorithms in C and creating hooks into our Python code or by simply writing the entire algorithm in C or C++.
%
%The clustering algorithm we used was a fairly straight forward flood fill implementation. We considered a cluster to be a contiguous region of pixels that registered a non-zero value. Thus, we iterated over each pixel in a frame and performed flood fill on any pixel that had not already been checked and had a non-zero value. The indices and value for each member of a contiguous region was then stored for later processing. This approach worked very well for the most part, except for when tracks from two different hits crossed, which caused two separate detector hits to be counted as one. This issue, shown in Figure~\ref{fig:crossedtracks}, occurred periodically throughout the flight but especially during ascent when cosmic ray intensity was at its peak. While this is a hard limitation of hybrid pixel detectors, a shorter frame exposure time could have prevented this issue during the high rate periods.
%
%
%
%\subsubsection{Cluster Sorting Algorithm and Morphological Analysis}
%
%Based on guidance from Dr.~George, who wrote his doctoral thesis on a hybrid pixel detector similar to the MiniPIX~\cite{stuartalgo}, we implemented a cluster sorting algorithm that sorts hits on our detector based on the pixel count, pixel density and the ratio between the length and width of the clusters bounding box. The algorithm we implemented, is nearly identical to the one developed by Dr.~George and outlined in Figure~\ref{fig:sortingalgo}. The only major difference is that, because of the thickness of the silicon chip of our MiniPIX detector, we had to increase the number of inner pixels from \num{4} to \num{8} to increase our heavy track discrimination. This is due to the fact that the original algorithm was tuned for a \SI{300}{\micro\meter} thick detector while our MiniPIX is \SI{500}{\micro\meter} thick, which resulted in thicker tracks. In general the morphological characteristics of a track left on the sensor can give us a good idea of the possible type of particle that hit the detector. For example a straight t rack on the detector is possibly explained as being deposited by a muon, fast proton or pion.
%
%	\begin{figure}[h!]
% 	\begin{center}
% 	\includegraphics[width=0.75\textwidth]{./Figures/stuartgraphic.pdf}
% 	\caption{Morphological characteristics used by the sorting algorithm that is based on clusters~\cite{stuartalgo}.}
% 	\label{fig:sortingalgo}
% 	\end{center}
% 	\end{figure}
%\newpage

%\subsection{Solar Radiation Methods}
%\label{UV Photodetector Testing}

%	\subsubsection{Responsitivity and Limitations}
%	The three photodiodes used for this experiment were the JIC 139 UV photodiodes with integrated amplifiers. We used the pin configuration provided by the datasheet to create an electronic circuit to integrate the photodiodes. To filter out noise, we included a \SI{3300}{\micro\farad} capacitor for each photodiode. With this configuration, we were allowed to apply a UV light source to record the measured Voltage output $V_{\text{Out}}$ with the RESU. The photocurrent $I_{\text{PD}}$ generated from incident light on the detector can be calculated by


%	\begin{equation}
%	I_{\text{PD}} = \frac{V_{\text{Out}}}{R_{f}}
%	\label{eq:photocurrent}
%	\end{equation}


%	where $R_{f}$ is the internal feedback resistor. Then, with the calibrated specifications provided, we were allowed to determine the irradiance of UV light $E_{e}$ by
	
%	\begin{equation}
%	E_{e} = \frac{I_{PD}}{(A \cdot R)}
%	\end{equation}
	

%	where ${A}$ is the size of the active area and ${R}$ is the spectral responsitivity. The specifications and internal components that make up the photodiode determine the peak irradiance or limit of saturation where the voltage supplied (\SI{5}{\volt}) is equal to the voltage out. We used this value in Equation~(\ref{eq:photocurrent}) to calculate a maximum irradiance of \SI{189}{\milli\watt\per\square\meter} as shown in Table~\ref{tab: UV-Specs}.


%	\begin{table}[h]
%	\centering
%	\caption{Specifications and peak irradiance of the UV photodiodes.}
%	\label{tab: UV-Specs}



%Results and Analysis
%Section: Flight Conditions and Environmental Data
\input{Results.tex} %RESU first
\input{Results_Rad.tex} %MiniPIX last
\newpage
\input{Results_Astro.tex} %Astro second

%\input{Results_UV.tex}
%\input{Simulations.tex} %Simulations results

%Discussion
\input{Discussion_RESU.tex} %RESU first
\input{Discussion_Astro.tex} %Astro second
\input{Discussion_Rad.tex} %MiniPIX last
\input{Discussion_Simulations.tex} %Simulations discussion

%Conclusion
\input{Conclusion.tex} 
\newpage
%Appendix
\input{Appendix.tex}
\newpage

%References
\newpage
%\section{References}
%\label{sec:References}

\begin{thebibliography}{9}
\bibitem{SORA}
S.A. Garcia Morelos, F.Brooks, S.Oliver, A.Walker, K.D. Portillo, R.B. Masek, D.Mroczek, D.Pena, J.Juarez, A.Cruz, D. Henandez, S.George, D. Pattison, A.L.Renshaw. \textit{Scientific Report for the UH Team.} SORA 2017 Mission Webpage. \url{http://laspace.lsu.edu/hasp/groups/Payload.php?py=2017&pn=10}.

\bibitem{silicon_sensor}
  MiniPIX - Miniaturized Portable USB Photon Counting Camera. (n.d.). Retrieved February 02, 2017, from \url{http://advacam.com/camera/minipix}.

\bibitem{LSU}
  Christner, B., Alleman, M., Bryan, N., Burke, S., Guzik, T.G., Granger, D., King, G. (2013) \textit{LSU HASP2013 PDF. Baton Rouge: Louisiana Space Consortium}.

 \bibitem{Solidworks}
   SolidWorks 3D CAD software \url{http://www.solidworks.com/}.
  
 \bibitem{COMSOL}
   COMSOL Multiphysics Simulation software \url{http://www.comsol.com}.

\bibitem{Extremophiles}
  Extremophiles \href{http://www.nytimes.com/2013/02/07/science/living-bacteria-found-deep-under-antarctic-ice-scientists-say.html}{http://www.nytimes.com/2013/02/07/science/living-bacteria-found-deep-\\under-antarctic-ice-scientists-say.html}.

\bibitem{canales}
 Canales D. C. and Ehteshami A., \textit{An attempt to sample atmospheric bacteria}, Houston, TX, 2015, January 11.

\bibitem{bexus}
Urbar, J., Scheirich, J., Jakubek, J., 2011. Medipix/Timepix cosmic ray tracking on BEXUS stratospheric balloonflights. Nucl. Instrum. Methods A 633, S206-209.
	
\bibitem{uv_irradiance}
  Calculating the UV Index. (2016, October 14). Retrieved June 03, 2017, from \url{https://www.epa.gov/sunsafety/calculating-uv-index-0}.

\bibitem{UVonDNA}
  Cockell CS, Horneck G., \textit{The history of the UV radiation climate of the earth--theoretical and space-based observations.}, Photochem Photobiol. 73(4):447-51 (2001).

\bibitem{cleanbox}
Clean box material \url{https://www.mcmaster.com/\#uhmw-polyethylene/=1aijn1p}.

\bibitem{valve}
Valve data sheet \url{http://www.generant.com/Literature/Series\%20VRV\%20Product\%20Literature.pdf}.

\bibitem{advacam}
  ADVACAM at \url{advacam.com}.

\bibitem{medipix}
  Medipix collaboration at \url{https://medipix.web.cern.ch/}.

\bibitem{JIC 139} 
  ELECTRO OPTICAL COMPONENTS, \textit{JIC 139}, Retrieved October, 17, 2017, from \url{inc.com/UV_detectors_silicon_carbide_photodiodes.html}.

\bibitem{Thor Labs} 
  Thor Labs, \textit{Unmounted UV Fused Silica Reflective ND Filters}. Retrieved August, 18, 2017, from \url{https://www.thorlabs.com/newgrouppage9.cfm?objectgroup_id=6106}.

\bibitem{RTL}
  RTL Genomics at \url{http://rtlgenomics.com/}.

\bibitem{lemon}
PCR Method at \url{http://www.journal-of-hepatology.eu/article/S0168-8278(17)32278-X/fulltext}.
	
\bibitem{illumina}
	Illumina, Inc. San Diego, California at \url{https://www.illumina.com/}

\bibitem{Microbial rRNA sequencing}
Microbial rRNA sequencing \url{http://pubs.rsc.org/En/content/articlehtml/2017/em/c6em00413j}.
  
\bibitem{stuartthesis} 22
  George, S., \textit{Dosimetric Applications of Hybrid Pixel Detectors}, University of Wollongong, Australia, 2015.

\bibitem{mpdatasheet}
  ADVACAM, \textit{MINIPIX Version 1.0 Datasheet}, Retrieved from \url{http://www.widepix.cz/files/datasheets/MiniPIX\%20v1.0\%20Datasheet.pdf}.

  \bibitem{numpy} 
  Numpy scientific computing library at \url{http://www.numpy.org/}.

\bibitem{stuartalgo}
	Reprinted from \textit{Dosimetric Applications of Hybrid Pixel Detectors} by Stuart George, 2015, Mixed Field Measurements and Cluster Formation with Timepix,52.

 \bibitem{regener}
 Regener E. \& Pfotzer G., \textit{Vertical Intensity of Cosmic Rays by Threefold Coincidences in the Stratosphere.}, Nature 136, 718-719, (1935). 

\bibitem{spaceballoonnetwork}
Space weather archives \url{http://spaceweather.com/archive.php?view=1&day=12&month=10&year=2016}.

\bibitem{dyastima}
  Paschalis, P. et al., \textit{Geant4 software application for the simulation of cosmic ray showers in the Earth? atmosphere. New Astronomy} {\bf 33} (2014):26.

\bibitem{geant4}
  Geant4 simulation toolkit at \url{http://geant4.cern.ch/}.


%\bibitem{mpjakubek}
%  Jan Jakubek, \textit{Precise energy calibration of pixel detector working in time-over-threshold mode} Institute of Experimental and Applied Physics, Czech Technical University in Prague, Czech Republic, 2011.
  

    


%\bibitem{magnetictool}
%  United States National Oceanic and Atmospheric Administration, \textit{Magnetic Field Calculators} [Data sets], Retrieved from \url{https://www.ngdc.noaa.gov/geomag-web/#igrfwmm}.

%\bibitem{gorman}
%	Gorman, J. (2013, February 06). \textit{Scientists Find Life in the Cold and Dark Under Antarctic Ice.} Retrieved September 15, 2016, from Scientists Find Life in the Cold and Dark Under Antarctic Ice.
	
 
 
%\bibitem{pumpsource}
%  \url{http://www.knfusa.com/?type=5600&amp;file=2079}.


%\bibitem{Horneck}
%  Horneck, G. 1993. The Biostack concept and its application in space and at accelerators: studies in Bacillus subtilis spores, p. 99-115. In C. E. Swenberg, G. Horneck, and E. G. Stassinopoulos (ed.), \textit{Biological effects and physics of solar and galactic cosmic radiation}[PDF], part A. Plenum Press, New York, NY. accessed 10/24/16  

%\bibitem{Horneck} 
%  Horneck, G. 2007. \textit{Space radiation biology}[PDF], p. 243-273. In E. Brinckmann (ed.), Biology in space and life on Earth. Wiley-VCH, Weinheim, Germany. Accessed 10/26/16

%\bibitem{Horneck}
%  Horneck, G., C. Baumstark-Khan, and G. Reitz. 2002. \textit{ Space microbiology: effects of ionizing radiation on microorganisms in space}[PDF], p. 2988-2996. In G. Bitton (ed.), The encyclopedia of environmental microbiology. John Wiley \& Sons, New York, NY. Accessed 10/30/16

%\bibitem{Horneck}
%  Horneck, G., C. Baumstark-Khan, and R. Facius. 2006. \textit{Radiation biology}[PDF], p. 292-335. In G. Cl?ment and K. Slenzka (ed.), Fundamentals of space biology. Kluwer Academic Publishers/Springer, Dordrecht, The Netherlands. accessed 11/4/16

%\bibitem{Kiefer}
%Kiefer, J., K. Schenk-Meuser, and M. Kost. 1996. \textit{Radiation biology}[PDF], p. 300-367. In D. Moore, P. Bie, and H. Oser (ed.), Biological and medical research in space. Springer, Berlin, Germany. accessed 11/9/16
 
	

\bibitem{SamURD}
	Alfonso Garcia Morelos, S. (2016, October 13).
	\textit{A Novel Microbe Trap.}
	Presentation at UH Undergraduate Research Day. \url{http://www.uh.edu/honors/undergraduate-research/}
	
	
%\bibitem{StevenURD}

%\bibitem{FreEttaWomensConference}

%\bibitem{StevenSchoolPres}

\bibitem{StevenURD}
  Oliver, S. J. (2017, October 12). 
  \textit{Stratospheric Organism and Radiation Analyzer}
  Presentation at UH Undergraduate Research Day. Retrieved October 12, 2017, from \url{http://www.uh.edu/honors/undergraduate-research/events/urday2017/}

\bibitem{StevenSchoolPres}
  Oliver, S. J. (2017, November 4). 
  \textit{STEM Life at UH}
  Presentation at UH Gathering of the Eagles STEM Symposium. \url{https://www.uh.edu/news-events/stories/2016/November/110416EaglesSTEM.php}

\bibitem{Fre}
  Brooks, F. (2017, January 14).
  \textit{Stratospheric Organism and Radiation Analyzer}
  Presentation at Rice University, APS Conferences for Undergraduate Women in Physics (CUiP). \url{http://www.google.com/url?q=http%3A%2F%2Fwww.aps.org%2Fprograms%2Fwomen%2Fworkshops%2Fcuwip.cfm&sa=D&sntz=1&usg=AFQjCNE5pImV-SVrb87CvgAa9RSfeCrYXg}  
  
  \bibitem{SamAPS}
	Alfonso Garcia Morelos, S. (2017, October 20).
	\textit{Stratospheric Organism and Radiation Analyzer}
	Retrieved October 20, 2017, from \textit{Bulletin of the American Physical Society}. \url{https://meetings.aps.org/Meeting/TSF17/Session/E5.3}
	
  
  \bibitem{MIT}
	MIT-Lemelson Award 2018.
	\url{https://lemelson.mit.edu/}
  

\end{thebibliography}%Bibliography file

%
\clearpage
%\bibliographystyle{SORA}
%\bibliography{SORA}
\end{document}
